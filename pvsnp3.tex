\documentclass[11pt]{article}
\usepackage{amsmath,amssymb,amsthm}
\usepackage{fullpage}
\usepackage{enumitem}

\begin{document}

\begin{center}
    {\LARGE \bf P vs NP Separation in the Universal Object Reference (UOR) Framework -- Part 3}
\end{center}
\vspace{1em}

\section*{1. Local vs Global Operations in UOR (No Efficient Global Search)}

\textbf{Theorem (No Polynomial-Time Global SAT Solution via Local UOR Operations).} \\
Within the UOR framework, no algorithm composed of a polynomial-length sequence of local elementary operations can, in general, solve an arbitrary SAT instance. In other words, there is no sequence of \(O(n^c)\) Clifford algebra or Lie group transformations that guarantees finding a satisfying assignment for all formulas of size \(n\) (unless a collapse like \(P=NP\) occurs, which we later reject). Any such sequence would necessarily leave exponentially many possibilities unexamined, precluding a guaranteed solution (pvnp2.pdf).

\textbf{Proof Sketch:} Each elementary UOR operation (as formally defined in the UOR model) has a localized effect on the state (pvnp2.pdf). For example, multiplying the state by a Clifford basis element \(e_j\) flips the \(j\)th bit of the assignment (acting like a NOT on that variable), or a Lie group action might swap or rotate a small set of components (pvnp2.pdf). Crucially, no single operation can simultaneously set or test an arbitrary large combination of bits --- it only affects a constant-bounded subset of the system (analogous to a single logic gate operating on a few bits). This locality means that after \(k\) such steps, at most \(k\) bits have been flipped (or a small set of bits toggled multiple times), and at best the process has explored on the order of \(2^k\) different partial assignments or states (pvnp2.pdf).

Now, consider a SAT instance with \(n\) boolean variables. There are \(N = 2^n\) possible assignments in the search space. A polynomial number of steps \(k = \mathrm{poly}(n)\) can explore at most \(2^k\) distinct assignments --- which for \(k = O(n^c)\) is still exponentially smaller than \(2^n\) when \(n\) is large (pvnp2.pdf). In fact, \(2^{O(n^c)} \ll 2^n\) for large \(n\), so any polynomial-time procedure can explicitly check only a vanishingly small fraction of all candidate solutions. Unless the SAT instance has some special structure that miraculously guides the algorithm to the correct assignment without extensive searching, a polynomial-time sequence will almost surely miss the satisfying assignment in the worst case. SAT being NP-complete means we can construct worst-case formulas with essentially arbitrary structure, thwarting any fixed greedy or local strategy (pvnp2.pdf). In such hard instances, satisfying one constraint tends to unsettle another, forcing a back-and-forth search that cannot be completed in polynomially many steps (pvnp2.pdf). The algorithm would have to explore a combinatorial number of possibilities to ensure all constraints are satisfied.

We formalize this intuition in UOR terms: the space of all possible assignments corresponds to a highly complex manifold of dimension \(n\) (one degree of freedom per variable). The satisfying states (assignments that make all clauses true) are discrete points within this space, typically exponentially numerous but isolated from each other by differences in many variable values (pvnp2.pdf). Any continuous path or sequence of local transformations from an initial incoherent state to a fully coherent (satisfying) state must decide on a value for each of the \(n\) variables at some stage (pvnp2.pdf). Each variable choice is essentially a binary decision, and there are \(n\) independent decisions to make; a local move can only resolve one (or a small group) of these at a time (pvnp2.pdf). Even if one tries a clever compound transformation (e.g. a higher-dimensional rotation in the Clifford/Lie space) to set many bits at once, UOR’s algebraic rules prevent simultaneously solving all constraints in one step --- each elementary generator from the Lie algebra or Clifford algebra has a limited, local action (pvnp2.pdf). In simple terms, UOR cannot “magically” coordinate an exponential number of constraint-satisfying decisions in polynomial time (pvnp2.pdf). The system must adjust the \(n\) degrees of freedom one by one or in small groups, and in the worst case that entails on the order of \(2^n\) trial combinations to get all variables correct (pvnp2.pdf).

Formally, assume toward contradiction that a polynomial-length sequence of local operations \(\hat S\) could always find a satisfying assignment for any SAT formula of size \(n\). By the universality of UOR operations, such a sequence \(\hat S\) would correspond to a polynomial-time algorithm in the classical sense (UOR can simulate any standard computation step-for-step (pvnp2.pdf)). In particular, \(\hat S\) could be used to decide SAT in polynomial time by starting from a blank assignment and applying \(\hat S\) to obtain a solution and then checking it (pvnp2.pdf). This would imply \(\mathit{SAT} \in P\), and hence \(P = NP\) since SAT is NP-complete (pvnp2.pdf). We are then forced to explain how \(\hat S\) avoids the combinatorial explosion of possibilities inherent in SAT. But as argued above, no such avoidance is possible: after polynomially many local steps, an exponential number of assignments remain unchecked (pvnp2.pdf). The only way \(\hat S\) could succeed is if it somehow encoded an exponential search implicitly --- effectively examining an exponential number of possibilities “all at once” via interference or symmetry. UOR, however, has no hidden non-local shortcut of that sort: any purported shortcut would require a highly entangled transformation that violates the local-action property of the allowed operations (pvnp2.pdf) or the coherence constraints of the system. In essence, a polynomial \(\hat S\) would need to encapsulate a full exponential decision tree within it (pvnp2.pdf), which contradicts the limited information processing capacity of a polynomial sequence of local transformations (pvnp2.pdf).

Therefore, \(\hat S\) cannot exist. No polynomial sequence of local UOR operations can globally collapse the SAT search space or reliably find a satisfying assignment for all instances. This establishes that solving SAT (and hence any NP-complete problem) requires, in the worst case, exponentially many elementary operations in the UOR framework (pvnp2.pdf).

\vspace{1em}

\section*{2. Explicit Coherence Norm Gradient and Stepwise Progress}

\textbf{Definition (Coherence Norm for SAT).} \\
In the UOR encoding of a Boolean formula (SAT instance), we associate an orthonormal basis element \(e_i\) (in a Clifford algebra fiber) to each clause \(C_i\) (pvnp2.pdf). We define the reference multivector 
\[
R = e_1 + e_2 + \cdots + e_m
\]
to represent the ideal state where all \(m\) clauses are satisfied (pvnp2.pdf). For a given assignment \(y\) (choice of truth values for all variables), we construct a corresponding multivector
\[
A_y = \sum_{i=1}^m v_i\, e_i,
\]
where \(v_i = 1\) if clause \(C_i\) is satisfied by \(y\), and \(v_i = 0\) if \(C_i\) is unsatisfied (pvnp2.pdf). In other words, \(A_y\) has a “1” in each clause direction that \(y\) satisfies. The coherence norm \(|\cdot|_c\) is the inner-product norm associated with an invariant inner product \(\langle\cdot,\cdot\rangle_c\) on these multivectors (with the \(e_i\) basis orthonormal) (pvnp2.pdf). We focus on the norm of the difference \(A_y - R\), which quantifies the assignment’s total constraint violation. One computes:
\[
\|A_y - R\|_c^2 \;=\; \langle A_y - R,\; A_y - R \rangle_c \;=\; \sum_{i=1}^m (v_i - 1)^2,
\]
using orthonormality \(\langle e_i, e_j\rangle_c = 0\) for \(i\neq j\) and \(\langle e_i,e_i\rangle_c=1\) (pvnp2.pdf). Since \((v_i-1)^2\) is \(0\) if \(v_i=1\) and \(1\) if \(v_i=0\), this simplifies to
\[
\|A_y - R\|_c^2 \;=\; \#\{i : v_i = 0\},
\]
the number of unsatisfied clauses under assignment \(y\) (pvnp2.pdf). Thus the squared coherence norm directly counts unsatisfied constraints, and the coherence norm 
\[
|A_y - R|_c \;=\; \sqrt{\text{(\# unsatisfied clauses)}},
\]
in particular, \(|A_y - R|_c = 0\) if and only if \(v_i=1\) for all \(i\), i.e. all clauses are satisfied (a fully coherent state) (pvnp2.pdf). This formalizes the idea that the coherence norm measures the “internal consistency” of the formula+assignment object (pvnp2.pdf) --- it is essentially an error metric or energy function for the SAT problem within UOR.

Now, consider how this coherence norm can change as we apply elementary operations to try to reach a satisfying assignment. We can view 
\[
F(y) := |A_y - R|_c^2
\]
as a non-negative function on the space of assignments (or more generally, on the continuous state manifold if we allow interpolating states). Solving SAT corresponds to finding some \(y\) with \(F(y)=0\). An algorithm that searches for a solution can be seen as attempting to decrease \(F\) step by step, by fixing unsatisfied clauses one by one until none remain (reaching the global minimum of \(F\)). We now formalize the gradient constraint on this process:

\bigskip

\textbf{Theorem (Gradient Bound on Coherence Norm Improvement).} \\
At most a constant or polynomially-bounded number of clause violations can be resolved by any single elementary UOR operation. Equivalently, there is a fixed upper bound \(\Delta_{\max}\) (independent of \(n\) exponential growth) on how much \(|A_y - R|_c^2\) can decrease in one step. In practice, each local operation can satisfy at most one additional clause (or a small set of clauses) that were previously unsatisfied (pvnp2.pdf). Thus, the coherence norm can only drop by a relatively small amount per step --- it cannot be minimized to \(0\) by any polynomial-length sequence of such small improvements unless the problem is easy (no conflicting constraints).

\bigskip

\textbf{Justification:} \\
An elementary UOR operation acts on a limited part of the assignment, such as flipping the value of a single variable or swapping a couple of bits (pvnp2.pdf). This operation will only directly affect the truth values of the clauses that depend on those particular variable(s). If, for example, we flip one variable \(x_j\) from false to true, all clauses \(C_i\) that contain \(x_j\) as a (non-negated) literal will potentially switch from unsatisfied to satisfied, and all clauses that contain \(\neg x_j\) (the negation) might switch from satisfied to unsatisfied. The net change in the number of satisfied clauses is bounded by the number of clauses that include \(x_j\). Let \(d_j\) be the number of clauses involving variable \(x_j\) (either positively or negatively). Then flipping \(x_j\) can affect at most \(d_j\) clauses. In the most favorable case, if \(x_j\) was previously the sole reason those \(d_j\) clauses were unsatisfied, the flip could satisfy up to \(d_j\) clauses at once (reducing \(F(y)\) by \(d_j\)). Typically \(d_j\) is not extremely large --- it is at most linear in the input size (each clause listing contributes to this count) and in many SAT formulations each variable appears in a limited number of clauses. More generally, any local move (even a compound Clifford/Lie operation) is restricted to a constant or poly-sized subsystem of the variables by design, so it cannot simultaneously fix an arbitrary large fraction of all clauses. No single operation can cause an exponential jump in the number of satisfied constraints; its influence is local. This is the formal meaning of the coherence norm having a gradual slope: one step traverses a small distance in the space of assignments, yielding at best a small drop in the “energy” \(F\) (pvnp2.pdf).

Moreover, because \(F(y)\) counts unsatisfied clauses, each clause constraint must be addressed at some point to reach \(F=0\). In the worst case, the clauses might be interdependent such that satisfying one clause causes another to become unsatisfied (a typical scenario in hard SAT instances) (pvnp2.pdf). In such cases, the algorithm might need to revisit clauses --- flipping one variable may fix clause \(C_i\) but spoil \(C_j\), and a later operation must fix \(C_j\) (possibly undoing \(C_i\) again). This kind of back-and-forth means the effective progress per operation can be even smaller than one clause at a time, as some steps are spent reversing earlier decisions. Nevertheless, even under optimistic assumptions (each step cleanly fixes one new clause without breaking others), one would need on the order of \(m\) steps to satisfy \(m\) clauses. Given \(m = \Theta(n)\) or another polynomial in input size for typical formulas, this suggests \(O(n)\) steps might suffice if there were no conflicts. However, for formulas that enforce an exponential search (for example, those requiring trying many combinations of variable assignments), the algorithm cannot maintain a monotonic decrease in \(F\) at the maximum rate for all steps. Eventually it must explore different branches of assignments, which entails some steps that do not decrease \(F\) or that increase it (when a previous choice is undone). The gradient descent metaphor thus breaks down: the landscape of \(F(y)\) has many local minima or plateaus that force exhaustive search rather than a straightforward downhill path.

We can formalize the limited improvement per step as follows. In the continuous viewpoint, let \(\nabla F\) be the gradient of \(F\) on the assignment manifold. Each elementary operation corresponds to moving the state in the direction of some generator \(g_i\) (a tangent direction in the Lie algebra or along a Clifford basis element). The directional derivative of \(F\) along any such direction \(g_i\) is determined by how \(g_i\) changes the clause satisfaction values \(v_i\). Because \(g_i\) acts only on certain variables, \(\nabla F\) projected onto that direction will be nonzero only for those clause terms depending on those variables. This means the steepest possible descent along any allowed operation is capped by the total “error” in the clauses involving that operation’s variables. In short, 
\[
\max_{\text{allowed step } g} \Bigl[F(\text{current}) - F(\text{after } g)\Bigr]
\]
is bounded by a constant or polynomial in \(n\). There is no direction in the state space that yields an exponential drop in \(F\) in one move, because that would require simultaneously resolving an exponential number of independent clause violations. UOR’s local operations do not permit such collective action (pvnp2.pdf).

Therefore, the coherence norm cannot be driven to zero in only polynomially many steps unless the instance happens to be exceptionally easy (e.g. each step miraculously fixes many clauses without conflict, which does not hold in general). Our earlier Theorem on Local vs Global operations already showed that a polynomial sequence can explore at most exponentially few assignments; here we add that even in terms of the “energy” landscape \(F(y)\), a polynomial-length trajectory can only reduce that energy by a polynomial amount --- insufficient to guarantee reaching the ground state for a hard instance. In fact, the hardness of SAT manifests as the presence of an exponentially large number of local minima or decision branches that an algorithm must navigate; the coherence norm’s gradient offers no huge shortcuts. This result aligns with and reinforces Theorem 2 above: finding a satisfying assignment (zero-norm state) from an arbitrary starting point generally requires an exponential sequence of small improvements (pvnp2.pdf). Each elementary operation yields at best a constant or slight reduction in incoherence, so polynomially many operations cannot accumulate enough total progress to reach full coherence in the worst case.

\vspace{1em}

\section*{3. Strengthening the NP = co-NP Argument (Unsatisfiability in UOR)}

Finally, we address the relationship between NP and co-NP in the UOR framework by considering certificates of unsatisfiability. Thus far, we have argued that finding a satisfying assignment is hard (requiring exponentially many local operations). We now show that proving no assignment exists (the co-NP task) is equally hard in UOR, which solidifies the conclusion that \(P \neq NP\).

\textbf{UOR Representation of an Unsatisfiability Certificate:} \\
In classical complexity, a certificate that a formula \(\Phi\) is unsatisfiable can be given by a proof (for instance, a resolution refutation) that derives a contradiction from \(\Phi\)’s clauses. Such a proof essentially shows that assuming a solution exists leads to an impossibility (like deriving both a clause and its negation, or deriving an empty clause). In UOR terms, an unsatisfiable formula means that for every assignment \(y\), the constructed multivector \(A_y\) differs from the ideal \(R\) in at least one clause component --- there is no state achieving full coherence (pvnp2.pdf). To certify unsatisfiability within UOR, one would need to demonstrate this fact using UOR operations or objects. For example, one might try to introduce an auxiliary sequence of transformations \(\{T_1, T_2, \dots, T_k\}\) acting on the formula+assignment representation that systematically accumulates the logical implications of the clauses and leads to a contradiction. This could be thought of as a UOR analog of a resolution proof: each operation combines two clause constraints (as resolution combines two clauses) or adds a new constraint component, with the goal of eventually reaching an inconsistent state (such as producing a clause basis element \(e_i\) that must be both present and absent, or producing a composite object that has a provably non-zero coherence norm no matter the assignment) (pvnp2.pdf). Another approach might be to construct a single invariant quantity or an expanded reference object that encapsulates all assignments, attempting to show that \(|A_y - R|_c\) has a lower bound away from 0 for all \(y\) (pvnp2.pdf). In any case, to prove a negative (no satisfying assignment), the system essentially has to either enumerate all possibilities or reason about all of them collectively.

\bigskip

\textbf{Theorem (NP \(\neq\) co-NP in UOR).} \\
Within the UOR model, there is no polynomial-time procedure --- using polynomially many UOR operations or a polynomial-size UOR object construction --- that can always certify a formula’s unsatisfiability. Any valid proof of unsatisfiability (global incoherence) in UOR must implicitly consider an exponential number of potential assignments or encode an exponential amount of information. Consequently, NP and co-NP are distinct classes in the UOR framework (pvnp2.pdf). In particular, assuming \(P=NP\) (and hence NP = co-NP) would imply the existence of polynomial unsatisfiability certificates, which contradicts this result. Therefore, we conclude \(P \neq NP\).

\textbf{Proof:} \\
Suppose, for the sake of contradiction, that unsatisfiability of any CNF formula could be certified with a UOR construction of only polynomial complexity (time or size). This would mean: given an unsatisfiable formula \(\Phi\) with \(n\) variables and \(m\) clauses, we can either (a) perform a sequence of at most \(\mathrm{poly}(n)\) UOR operations that detect a contradiction and confirm no assignment works, or (b) construct a polynomial-size UOR object that serves as a certificate encapsulating the inconsistency of \(\Phi\). In either case, UOR would be able to decide unsatisfiability in polynomial time, implying co-NP \(\subseteq\) P (since the certificate can be verified efficiently) and thus NP = co-NP (if we assumed \(P = NP\) or even just NP \(\subseteq\) P).

However, our earlier results indicate why this is implausible. Proving unsatisfiability requires showing every possible assignment fails. In a UOR proof, this means showing that for each possible \(y\), \(\|A_y - R\|_c^2 \ge 1\) (at least one clause is unsatisfied). How can a polynomial process establish this for all \(2^n\) assignments? A deterministic procedure would effectively have to either check each assignment or use logical deductions to rule out large sets of assignments at once. The latter approach is exactly what a classical proof system like resolution does: it incrementally derives constraints that eliminate many assignments in one step. But it’s known in computational logic that certain families of formulas have no short proofs. For example, pigeonhole principle formulas (which state that \(n+1\) pigeons cannot fit into \(n\) holes) are unsatisfiable, but any resolution proof of this fact has exponential length (pvnp2.pdf). This is a specific case of a broader belief: co-NP does not have polynomial-size proofs in general (i.e. NP \(\neq\) co-NP). Our assumption contradicts these results.

Translating this to UOR: a polynomial sequence of UOR operations trying to derive a contradiction from \(\Phi\) would mirror a polynomial-length logical refutation. For arbitrary hard formulas, no such refutation exists (pvnp2.pdf). The UOR operations --- each being local and combining only a few constraints at a time --- cannot magically compress an exponential logical argument into polynomial size (pvnp2.pdf). To rule out all assignments, one inevitably needs to either enumerate them or use an argument that implicitly covers exponentially many cases. Any general method to do the latter in polynomial time would imply a breakthrough in proof complexity (polynomial-length proofs for all tautologies/contradictions), which is considered unlikely and is unsupported by decades of evidence in complexity theory. Even if UOR allows adding extra “dimensions” or ancillary objects to help combine constraints, one would eventually need to encode a condition that differentiates each of the exponentially many assignments (for instance, a product term or sum term for each assignment (pvnp2.pdf)). Such an encoding inevitably blows up in size or requires exponentially many steps to construct.

Therefore, unsatisfiability cannot be certified in polynomial time within UOR. This implies that co-NP (the class of unsatisfiable formulas) is not contained in the class of problems decidable by polynomial-length UOR procedures. In combination with our earlier conclusion that NP problems (like SAT) also cannot be solved in polynomial-length sequences (NP \(\not\subseteq\) P in UOR), we conclude NP \(\neq\) co-NP in the UOR framework (pvnp2.pdf). Finally, since a hypothetical \(P=NP\) would entail NP = co-NP (as \(P\) is closed under complementation in the classical sense) (pvnp2.pdf), our result directly forces \(P \neq NP\). In summary, the UOR framework, when rigorously applied, does not permit an efficient solution to NP-complete problems nor a short certificate for unsatisfiable instances --- fully consistent with the classical belief that \(P \neq NP\). \(\blacksquare\)

\vspace{1em}

\noindent\textbf{Sections Included:}
\begin{enumerate}[label=\arabic*.]
    \item Local vs Global Operations in UOR (No Efficient Global Search)
    \item Explicit Coherence Norm Gradient and Stepwise Progress
    \item Strengthening the NP = co-NP Argument (Unsatisfiability in UOR)
\end{enumerate}

\end{document}