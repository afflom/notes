\documentclass{article}
\usepackage{amsmath,amsthm,amssymb}
\usepackage{geometry}
\geometry{margin=1in}
\usepackage{enumerate}

\newtheorem{theorem}{Theorem}[section]
\newtheorem{lemma}[theorem]{Lemma}
\newtheorem{definition}[theorem]{Definition}

\begin{document}

\title{Deriving Analytic Number Theory Results in the Prime Framework}
\author{The UOR Foundation}
\date{\today}
\maketitle

\begin{abstract}
  In the Prime Framework natural numbers are constructed intrinsically by embedding them into a fiber algebra, and each number factors uniquely into intrinsic primes. Together with the construction and spectral analysis of the Prime Operator, an intrinsic zeta function
  \[
  \zeta_{\mathrm{P}}(s) = \frac{1}{D(s)} = \prod_{p\,\text{intrinsic}} \frac{1}{1-p^{-s}},
  \]
  is obtained. In this paper we show that classical analytic techniques --- such as Mellin inversion, contour integration, and Tauberian arguments --- can be applied entirely within the framework to derive the Prime Number Theorem and explicit formulas for both the prime-counting function \(\pi(X)\) and the \(n\)th prime \(p_n\). Moreover, the symmetry inherent in the axioms forces \(\zeta_{\mathrm{P}}(s)\) to satisfy a functional equation and an internal analogue of the Riemann Hypothesis.
\end{abstract}

\section{Introduction}
Within the Prime Framework the natural numbers are not assumed a priori but are constructed via an intrinsic embedding into a fiber algebra \(C_x\) over a smooth manifold \(M\). Each natural number \(N\) is represented by a canonical element \(\widehat{N}\) whose graded components encode its digit expansions in every base \(b\ge2\). This embedding leads to a definition of \emph{intrinsic primes} and the unique factorization of numbers analogous to the classical fundamental theorem of arithmetic.

Concurrently, the divisor structure of natural numbers is captured by the linear operator \(H\) on the Hilbert space \(\ell^2(\mathbb{N})\) defined by
\[
H(\delta_N) = \sum_{d\,\mid\,N} \delta_d,
\]
where \(\{\delta_N\}_{N\in\mathbb{N}}\) is the standard orthonormal basis. Spectral analysis of \(H\) yields a formal determinant \(D(u)=\det(I-uH)\) which, by the unique factorization property, factorizes into an Euler product. With the substitution \(u=p^{-s}\) one defines the intrinsic zeta function
\[
\zeta_{\mathrm{P}}(s)=\frac{1}{D(s)}=\prod_{p\,\text{intrinsic}} \frac{1}{1-p^{-s}},
\]
recovering the classical Euler product for \(\Re(s)>1\).

\section{Deriving the Prime Number Theorem}
With the intrinsic zeta function \(\zeta_{\mathrm{P}}(s)\) defined internally, one may apply classical analytic techniques. In particular, the Mellin inversion formula allows us to express the prime-counting function as
\[
\pi(X) = \frac{1}{2\pi i} \int_{c-i\infty}^{c+i\infty} \frac{X^s}{s\,\zeta_{\mathrm{P}}(s)}\, ds, \quad c>1.
\]
Since \(\zeta_{\mathrm{P}}(s)\) has a simple pole at \(s=1\) (corresponding to a zero of \(1/\zeta_{\mathrm{P}}(s)\)), the contour may be shifted leftwards. By applying the residue theorem and a Tauberian argument, the dominant contribution comes from the residue at \(s=1\), yielding
\[
\pi(X) \sim \frac{X}{\ln X} \quad \text{as } X \to \infty.
\]
This establishes the Prime Number Theorem within the axiomatic system.

\section{Explicit Formulas for \(\pi(X)\) and \(p_n\)}
A more refined analysis through contour integration provides an explicit formula for \(\pi(X)\). Let \(\operatorname{Li}(X) = \int_2^X \frac{dt}{\ln t}\) denote the logarithmic integral. By accounting for the contributions of the nontrivial zeros \(\rho\) of \(\zeta_{\mathrm{P}}(s)\), one obtains
\[
\pi(X) = \operatorname{Li}(X) - \sum_{\rho} \operatorname{Li}(X^{\rho}) + R(X),
\]
where \(R(X)\) represents lower order correction terms. Inverting the relation \(\pi(p_n)=n\) then leads to an explicit asymptotic expansion for the \(n\)th prime:
\[
p_n = n\ln n + n\ln\ln n - n + O\!\left(\frac{n}{\ln n}\right).
\]

\section{Functional Equation and the Riemann Hypothesis Analogue}
The symmetry axioms of the Prime Framework, particularly the invariance under the action of the symmetry group \(G\) and the coherence imposed by the inner product, enforce a duality in the spectral properties of \(H\). As a result, the intrinsic zeta function \(\zeta_{\mathrm{P}}(s)\) satisfies a functional equation of the form
\[
\zeta_{\mathrm{P}}(s) = \Phi(s)\, \zeta_{\mathrm{P}}(1-s),
\]
where \(\Phi(s)\) is an explicitly determined factor (typically involving Gamma functions, powers of \(\pi\), and sine functions) that arises solely from the internal structure of the framework. Furthermore, by showing that the Prime Operator \(H\) (or an appropriately normalized variant) is similar to a self-adjoint operator, one deduces that all nontrivial zeros of \(\zeta_{\mathrm{P}}(s)\) lie on the critical line \(\Re(s)=\frac{1}{2}\). This internal result constitutes an analogue of the Riemann Hypothesis within the Prime Framework.

\section{Conclusion}
By embedding natural numbers into a fiber algebra and leveraging their unique factorization into intrinsic primes, the Prime Framework provides a self-contained setting in which the Prime Operator \(H\) encapsulates the divisor structure of numbers. The spectral analysis of \(H\) leads to an intrinsic zeta function \(\zeta_{\mathrm{P}}(s)\) whose Euler product representation recovers classical analytic number theory results. Employing techniques such as Mellin inversion, contour integration, and Tauberian arguments within this axiomatic system, we have derived the Prime Number Theorem as well as explicit formulas for \(\pi(X)\) and \(p_n\). Additionally, the symmetry of the framework guarantees that \(\zeta_{\mathrm{P}}(s)\) satisfies a functional equation and an internal version of the Riemann Hypothesis. This self-contained derivation highlights the robustness and internal consistency of the Prime Framework in establishing key results of analytic number theory from first principles.

\end{document}
