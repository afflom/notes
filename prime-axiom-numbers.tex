\documentclass[12pt]{article}
\usepackage{amsmath,amssymb,amsthm}
\usepackage{geometry}
\geometry{margin=1in}

\newtheorem{axiom}{Axiom}[section]
\newtheorem{theorem}{Theorem}[section]
\newtheorem{lemma}{Lemma}[section]
\newtheorem{definition}{Definition}[section]
\newtheorem*{remark}{Remark}

\title{Intrinsic Embedding of Numbers in the Prime Framework}
\author{The UOR Foundation}
\date{\today}

\begin{document}
\maketitle

\section{Introduction}
In the Prime Framework, numbers are not assumed to exist a priori; instead, they are constructed internally by encoding their representations in every possible base. This document presents a self-contained proof of the \emph{Universal Number Embedding}: for each natural number \(N\), there exists a unique canonical element \(\hat{N}\) in the fiber algebra \(C_x\) (typically a Clifford algebra) whose graded components carry the digit sequences for each base. The coherence inner product on \(C_x\) minimizes any internal discrepancies, ensuring that the resulting embedding is both unique and fully consistent.

\section{Preliminaries}
We assume the following components from the Prime Framework's Axiomatic Foundation:
\begin{enumerate}
  \item \textbf{Reference Manifold:} \(M\) is a smooth, connected, orientable manifold equipped with a nondegenerate metric \(g\).
  \item \textbf{Algebraic Fibers:} For each point \(x\in M\), there exists an associative algebra \(C_x\) (typically a Clifford algebra) given by
  \[
  C_x := \mathrm{Cl}(T_xM, g_x),
  \]
  where \(T_xM\) is the tangent space at \(x\) and \(g_x\) is the quadratic form induced by \(g\).
  \item \textbf{Symmetry Group Action:} A Lie group \(G\) acts by isometries on \(M\) and lifts to each fiber \(C_x\) via isomorphisms
  \[
  \Phi(h)_x : C_x \to C_{h\cdot x}, \quad \forall\, h\in G,\; x\in M,
  \]
  ensuring consistency of local representations under transformations.
  \item \textbf{Coherence Inner Product:} Each \(C_x\) is endowed with a \(G\)-invariant, positive-definite inner product \(\langle \cdot,\cdot\rangle_c\) which induces the norm
  \[
  \|a\|_c = \sqrt{\langle a,a \rangle_c}.
  \]
  This inner product is designed so that if an abstract object is represented in multiple ways within \(C_x\), any discrepancy increases the norm. Hence, the unique minimal-norm element represents the fully coherent embedding.
\end{enumerate}

\section{Universal Number Embedding}
\begin{definition}[Universal Number Embedding]
For each natural number \(N \in \mathbb{N}\), express \(N\) in every possible base \(b \ge 2\) as
\[
N = a_k(b)\,b^k + a_{k-1}(b)\,b^{k-1} + \cdots + a_1(b)\,b + a_0(b),
\]
where \(0 \le a_i(b) < b\) are the digits of \(N\) in base \(b\). Define the universal embedding of \(N\) as the collection
\[
\mathcal{E}(N) := \left\{ \big(a_0(b), a_1(b), a_2(b), \dots \big)_b \;:\; b=2,3,4,\dots \right\}.
\]
In the fiber \(C_x\) at a fixed reference point \(x\in M\), we construct an element \(\hat{N} \in C_x\) whose different graded components are designated to carry the digit sequences of \(N\) in each base. This mapping,
\[
N \mapsto \hat{N},
\]
is the intrinsic embedding of \(N\) into the framework.
\end{definition}

\section{Canonical Representation via Coherence Norm Minimization}
Let \(S_N \subset C_x\) be the set of all elements that encode the number \(N\) according to the universal embedding \(\mathcal{E}(N)\); that is,
\[
S_N = \{\, a \in C_x : a \text{ encodes the base--\(b\) digit sequences of } N \text{ for all } b\ge2 \,\}.
\]
The coherence inner product \(\langle\cdot,\cdot\rangle_c\) on \(C_x\) assigns to each \(a\in S_N\) a norm \(\|a\|_c\) which increases if the digit representations in different graded components are inconsistent. Thus, the \emph{canonical representation} of \(N\) is the unique element \(\hat{N}\in S_N\) that minimizes \(\|a\|_c\).

\begin{theorem}[Existence and Uniqueness of the Canonical Embedding]
For every natural number \(N\in\mathbb{N}\), there exists a unique (up to the action of the symmetry group \(G\)) element \(\hat{N} \in C_x\) that minimizes the coherence norm over \(S_N\). This element \(\hat{N}\) is the canonical intrinsic embedding of \(N\) in the Prime Framework.
\end{theorem}

\begin{proof}
\textbf{Existence:}\\
Since \(S_N\) is nonempty (a trivial representation can be constructed by encoding \(N\) in a designated base and replicating it consistently in all graded components), and because the constraints ensuring that the digit sequences in each base represent the same number \(N\) are linear, the set \(S_N\) is a closed subset of the finite-dimensional vector space \(C_x\). The function
\[
f : S_N \to \mathbb{R}_{\ge 0}, \quad f(a) = \|a\|_c,
\]
being continuous, attains its minimum on \(S_N\) by the extreme value theorem. Denote by \(\hat{N}\) an element in \(S_N\) where the minimum is achieved.

\textbf{Uniqueness:}\\
Assume that there exist two distinct minimal representations \(\hat{N}_1, \hat{N}_2 \in S_N\) such that
\[
\|\hat{N}_1\|_c = \|\hat{N}_2\|_c = m.
\]
Consider the average
\[
a = \frac{1}{2}(\hat{N}_1 + \hat{N}_2).
\]
Since the property of encoding \(N\) is linear, \(a\) also belongs to \(S_N\). However, by the strict convexity of norms induced by an inner product, we have
\[
\|a\|_c < \frac{1}{2}\|\hat{N}_1\|_c + \frac{1}{2}\|\hat{N}_2\|_c = m,
\]
which contradicts the minimality of \(m\). Thus, the minimal representation is unique.

\textbf{Equivariance under \(G\):}\\
If \(\hat{N}\) is the canonical representation at a point \(x\in M\) and \(y\in M\) is any other point, then by the symmetry axiom there exists \(h\in G\) such that \(h\cdot x = y\) and
\[
\|\Phi(h)_x(\hat{N})\|_c = \|\hat{N}\|_c.
\]
By uniqueness at the fiber \(C_y\), it follows that \(\Phi(h)_x(\hat{N})\) is the canonical representation at \(y\). Hence, the embedding is unique up to the action of \(G\).

This completes the proof.
\end{proof}

\section{Conclusion}
We have demonstrated that within the Prime Framework, each natural number \(N\) is intrinsically embedded as an element \(\hat{N}\in C_x\) whose graded components simultaneously encode all positional representations of \(N\). The coherence inner product forces any discrepancies between these representations to increase the norm, and thus the unique minimal-norm element provides a canonical embedding of \(N\). This construction creates the natural numbers internally and serves as the foundational step for developing further number-theoretic concepts from first principles.

\end{document}
