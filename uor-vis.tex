\documentclass[11pt]{article}
\usepackage{amsmath,amssymb,algorithm,algpseudocode}
\usepackage{geometry}
\geometry{margin=1in}

\begin{document}

\section*{The Prime Algorithm Function}

The function signature is:
\[
\texttt{PrimeAlgorithm} : \mathbb{N} \to \{\mathtt{PRIME},\mathtt{COMPOSITE}\}
\]
It takes a natural number \(N\) as input and returns either \(\mathtt{PRIME}\) or \(\mathtt{COMPOSITE}\).

Below is the complete pseudocode:

\begin{algorithm}
\caption{\textsc{PrimeAlgorithm}(\(N\))}
\begin{algorithmic}[1]
\Require \(N \ge 2\) is a natural number (encoded in the UOR/Clifford framework)
\Ensure Returns \(\mathtt{PRIME}\) if \(N\) is prime; returns \(\mathtt{COMPOSITE}\) otherwise.

\State Compute a small bound \(r\), e.g. \(r \gets \lceil (\log_2 N)^2 \rceil\)
\For{\(a = 2\) to \(r\)}
    \If{\(\gcd(a,N) \neq 1\)}
        \State \Return \(\mathtt{COMPOSITE}\) \Comment{A nontrivial factor found}
    \EndIf
\EndFor

\Comment{Perform the polynomial congruence test (a variant of the AKS test)}
\For{\(a = 1\) to \(r\)}
    \State Compute the polynomial \(P_a(X) \gets (X+1)^N \mod (N, X^r-1)\)
    \State Compute \(Q_a(X) \gets X^N + 1 \mod (N, X^r-1)\)
    \If{\(P_a(X) \not\equiv Q_a(X)\)}
        \State \Return \(\mathtt{COMPOSITE}\) \Comment{Congruence fails, \(N\) is composite}
    \EndIf
\EndFor

\State \Return \(\mathtt{PRIME}\)
\end{algorithmic}
\end{algorithm}

\end{document}
