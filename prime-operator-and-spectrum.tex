\documentclass{article}
\usepackage{amsmath,amsthm,amssymb}
\usepackage{geometry}
\geometry{margin=1in}
\usepackage{enumitem}

\newtheorem{theorem}{Theorem}[section]
\newtheorem{definition}[theorem]{Definition}

\begin{document}

\title{Constructing the Prime Operator and Analyzing Its Spectrum\\ in the Prime Framework}
\author{The UOR Foundation}
\date{\today}
\maketitle

\begin{abstract}
  In the Prime Framework natural numbers are constructed intrinsically by embedding them into a fiber algebra, and every number factors uniquely into intrinsic primes. In this paper we construct a linear operator \(H\) on a Hilbert space which encodes the divisor structure of numbers and use its spectral properties to recover the Euler product formula for the Riemann zeta function. Our treatment is self-contained and relies solely on the axiomatic foundation of the Prime Framework.
\end{abstract}

\section{Introduction}
In the Prime Framework the natural numbers are not assumed to exist a priori but are obtained via an intrinsic embedding into an algebra \(C_x\) at a point \(x\) of a smooth manifold \(M\). Each number \(N\) is represented by an element \(\widehat{N}\) whose graded components encode its digit expansions in every base \(b\ge2\). The coherence inner product on \(C_x\) guarantees that each such embedded number has a unique representation and factors uniquely into \emph{intrinsic primes}. In what follows, we construct a linear operator that reflects the divisor structure underlying this unique factorization, and we analyze its spectrum to derive the Euler product for the zeta function.

\section{Preliminaries}
We assume the following core axioms of the Prime Framework:
\begin{enumerate}[label=\textbf{Axiom \arabic*:},leftmargin=2cm]
  \item \textbf{Reference Manifold:} There exists a smooth, connected, and orientable manifold \(M\) with a nondegenerate metric \(g\).
  \item \textbf{Algebraic Fibers:} For each \(x\in M\) there is an associative algebra \(C_x\) (typically a Clifford algebra) that carries the local algebraic structure.
  \item \textbf{Symmetry Group Action:} A Lie group \(G\) acts by isometries on \(M\) and lifts to \(C_x\) via algebra isomorphisms.
  \item \textbf{Coherence Inner Product:} Each \(C_x\) is equipped with a positive-definite, \(G\)-invariant inner product \(\langle \cdot,\cdot \rangle_c\) with induced norm \(\|a\|_c=\sqrt{\langle a,a\rangle_c}\). This inner product forces all representations of the same number to cohere, thereby selecting a unique minimal-norm (canonical) embedding.
\end{enumerate}

\subsection{Intrinsic Embedding and Unique Factorization}
Each natural number \(N\) is embedded in \(C_x\) via its \emph{universal number embedding}. By expressing
\[
N = a_k(b) b^k + a_{k-1}(b) b^{k-1} + \cdots + a_0(b)
\]
in every base \(b\ge2\) and encoding these digit sequences in distinct graded components of \(C_x\), the coherence inner product singles out the unique minimal-norm representation \(\widehat{N}\). In this framework, an embedded number \(\widehat{N}\) (with \(N>1\)) is defined to be \emph{intrinsic prime} if whenever 
\[
\widehat{N} = \widehat{A} \cdot \widehat{B},
\]
for \(\widehat{A},\widehat{B} \in C_x\) corresponding to natural numbers \(A\) and \(B\), one of \(A\) or \(B\) equals \(1\). It is then proved that every embedded number factors uniquely into intrinsic primes, mirroring the classical fundamental theorem of arithmetic.

\section{Construction of the Prime Operator}
We now define a linear operator \(H\) on the Hilbert space \(\ell^2(\mathbb{N})\) which encodes the divisor structure inherent in unique factorization.

\subsection{Definition of \(H\)}
Let \(\{\delta_N : N\in\mathbb{N}\}\) be the standard orthonormal basis for \(\ell^2(\mathbb{N})\). Define the linear operator \(H : \ell^2(\mathbb{N}) \to \ell^2(\mathbb{N})\) by
\[
H(\delta_N) = \sum_{d\,\mid\,N} \delta_d,
\]
where the sum runs over all positive divisors \(d\) of \(N\). In other words, for any \(f\in \ell^2(\mathbb{N})\) we have
\[
(Hf)(N) = \sum_{d\,\mid\,N} f(d).
\]

\subsection{Properties of \(H\)}
\begin{itemize}
  \item \textbf{Boundedness:} Since each natural number \(N\) has finitely many divisors, the sum defining \(H(\delta_N)\) is finite. Standard estimates on the divisor function ensure that \(H\) is a bounded operator on \(\ell^2(\mathbb{N})\).
  \item \textbf{Self-Adjointness:} The operator \(H\) is self-adjoint with respect to the standard inner product on \(\ell^2(\mathbb{N})\). This follows from the symmetry of the divisor relation; indeed, if \(d\mid N\) then \(d\le N\), and the matrix representation of \(H\) has real entries satisfying \(H_{N,d} = H_{d,N}\) when both are nonzero.
  \item \textbf{Positivity:} For any \(f\in\ell^2(\mathbb{N})\), one can show that \(\langle f, Hf \rangle \ge 0\), so \(H\) is a positive operator.
\end{itemize}

\section{Spectral Analysis and the Euler Product}
We now turn to the spectral analysis of \(H\) by forming its formal determinant and relating it to the intrinsic primes.

\subsection{The Formal Determinant}
Define the formal determinant
\[
D(u) = \det(I - u H),
\]
where \(u\) is a complex parameter and \(I\) is the identity operator on \(\ell^2(\mathbb{N})\). By expanding the logarithm of the determinant we have
\[
\log D(u) = -\sum_{k\ge1} \frac{u^k}{k} \, \mathrm{Tr}(H^k),
\]
where the trace \(\mathrm{Tr}(H^k)\) counts the number of ways to express a natural number as a chain of divisors of length \(k\).

\subsection{Factorization via Unique Factorization}
The unique factorization property established in the Prime Framework implies that every natural number \(N\) factors uniquely into intrinsic primes. Consequently, the traces \(\mathrm{Tr}(H^k)\) decompose multiplicatively over the contributions of these intrinsic primes. In the formal determinant \(D(u)\), this decomposition leads to a factorization of the form
\[
D(u) = \prod_{p\,\text{intrinsic}} \det\Bigl(I - u\, H^{(p)}\Bigr),
\]
where \(H^{(p)}\) denotes the restriction of \(H\) to the invariant subspace corresponding to powers of the intrinsic prime \(p\). An analysis on each such subspace (which is isomorphic to a space spanned by \(\{\delta_{p^k} : k\ge0\}\)) shows that
\[
\det\Bigl(I - u\, H^{(p)}\Bigr) = 1 - u.
\]
Thus, we obtain
\[
D(u) = \prod_{p\,\text{intrinsic}} (1 - u).
\]

\subsection{Recovering the Euler Product}
By choosing the substitution
\[
u = p^{-s},
\]
the formal determinant becomes a function of \(s\). More precisely, define
\[
D(s) := \det\Bigl(I - p^{-s} H\Bigr).
\]
Then the above factorization implies
\[
D(s) = \prod_{p\,\text{intrinsic}} \bigl(1 - p^{-s}\bigr).
\]
Taking the reciprocal, we define the intrinsic zeta function by
\[
\zeta_{\mathrm{P}}(s) := \frac{1}{D(s)} = \prod_{p\,\text{intrinsic}} \frac{1}{1 - p^{-s}},
\]
which, for \(\Re(s)>1\), is exactly the Euler product representation of the Riemann zeta function. This derivation is completely internal to the Prime Framework and follows solely from its axioms and the unique factorization of intrinsic primes.

\section{Conclusion}
We have constructed a linear operator \(H\) on \(\ell^2(\mathbb{N})\) by
\[
H(\delta_N)=\sum_{d\,\mid\,N} \delta_d,
\]
which encapsulates the divisor structure of natural numbers as embedded in the Prime Framework. By analyzing the formal determinant \(D(u)=\det(I-uH)\) and invoking the unique factorization property of intrinsic primes, we have shown that
\[
\zeta_{\mathrm{P}}(s) = \frac{1}{D(s)} = \prod_{p\,\text{intrinsic}} \frac{1}{1-p^{-s}},
\]
recovering the Euler product for the Riemann zeta function for \(\Re(s)>1\). This self-contained proof illustrates how the spectral properties of the Prime Operator reflect the fundamental arithmetic structure encoded within the axiomatic system.

\end{document}
