\documentclass[11pt]{article}
\usepackage[margin=1in]{geometry}
\usepackage{amsmath,amssymb,amsthm,amsfonts}
\usepackage{hyperref}
\usepackage{enumitem}
\usepackage{bm}
\usepackage{cite}

\newtheorem{theorem}{Theorem}[section]
\newtheorem{definition}[theorem]{Definition}
\newtheorem{proposition}[theorem]{Proposition}
\newtheorem{remark}[theorem]{Remark}
\newtheorem{lemma}[theorem]{Lemma}
\newtheorem{example}[theorem]{Example}

\begin{document}

%=========================================================
% TITLE AND ABSTRACT
%=========================================================

\title{\textbf{Object-Oriented Ontology (OOO) Quadruple Object in the \\
Universal Object Reference (UOR) Framework}\\[5pt]
\large Main Text with Supplement}
\author{}
\date{}
\maketitle

\begin{abstract}
This document unifies two complementary discussions:

\textbf{(A)} A primary exposition on the \emph{Object-Oriented Ontology (OOO)} notion of the 
quadruple object, recast in the \emph{Universal Object Reference (UOR)} framework.

\textbf{(B)} A supplement addressing specific technical points about base-\(b\) expansions, 
commutator norms, and gauge symmetries relevant to the OOO quadruple structure.

The combined analysis demonstrates how OOO's four poles 
(real object, sensual object, real qualities, sensual qualities) 
can be formulated in a noncommutative Clifford-algebraic system, 
with partial overlaps and commutators measuring “withdrawal” 
and relational interplay.
\end{abstract}

\tableofcontents

%=========================================================
% MAIN TEXT: OOO QUADRUPLE OBJECT IN UOR
%=========================================================

\section{Introduction}
\label{sec:intro}

\subsection{Overview of Object-Oriented Ontology and the Quadruple Object}
Object-Oriented Ontology interprets \emph{objects} as primary ontological entities 
whose “withdrawn” core cannot be entirely reduced to human observation or 
to their relations. In Graham Harman's formulation, each object is said to have 
four aspects or “poles”:
\begin{itemize}[itemsep=0pt]
\item \textbf{Real Object (RO)}: the inaccessible essence,
\item \textbf{Sensual Object (SO)}: the appearance or “phenomenal” side,
\item \textbf{Real Qualities (RQ)}: features belonging to the object’s essence,
\item \textbf{Sensual Qualities (SQ)}: features that appear or vanish 
      under specific observational or relational conditions.
\end{itemize}
These four aspects are interrelated yet never collapse into a single, 
fully transparent entity.

\subsection{Universal Object Reference (UOR) Framework}
The Universal Object Reference (UOR) framework proposes that all definable objects 
and structures can be embedded into a finite-dimensional Clifford algebra (or a similar 
$*$-algebra). Key elements include:
\begin{itemize}[itemsep=0pt]
\item \textbf{Stable idempotents} and prime irreducibles to represent “fundamental seeds,”
\item \textbf{Base-$b$ expansions} that discretize coordinates or states at multiple scales,
\item \textbf{Partial noncommutativity} capturing how certain objects or aspects 
      do not commute (thus indicating “withdrawal” or irreducibility),
\item \textbf{Possible gauge symmetries} or spin group actions 
      that reorder subalgebra blocks.
\end{itemize}
In this framework, each “object” can be identified with certain projections or 
idempotents in the algebra, along with transformations in subalgebras that represent 
relational or observational vantage points.

\subsection{Document Outline}
Sections \ref{sec:quadObj}–\ref{sec:gauge} constitute the main exposition, 
presenting a rigorous definition for the Quadruple Object within UOR, 
and discussing “withdrawal” via commutators, partial overlaps, 
and possible gauge-structured subalgebras for real vs.\ sensual qualities. 
Then, Section \ref{sec:supplement} contains the supplement that addresses 
specific feedback about base-$b$ expansions, explicit commutator norms, 
and additional symmetries.

%=========================================================
% MAIN TEXT: DEFINING THE QUADRUPLE OBJECT
%=========================================================

\section{Defining the Quadruple Object in UOR}
\label{sec:quadObj}

\subsection{UOR Algebraic Basics}
Let \(\mathcal{A}\) be a unital $*$-algebra, e.g.\ a finite-dimensional Clifford algebra 
augmented by digit operators and possible gauge factors. In many UOR treatments:
\begin{itemize}[itemsep=0pt]
\item \textbf{Stable idempotents} $e\in \mathcal{A}$ represent “object seeds” 
      that cannot be further decomposed under small perturbations,
\item \textbf{Noncommutative geometry} ensures not all elements commute, 
      reflecting that objects cannot be fully “transparent” to every subalgebra,
\item \textbf{Digit expansions} $P_{d,k}$ provide a fractal or multi-scale 
      approach to organizing states or coordinates.
\end{itemize}

\subsection{Formal Quadruple Object}
\begin{definition}[Quadruple Object in UOR]
A \emph{quadruple object} $Q\in \mathcal{A}^4$ is given by 
\[
Q = (\mathbf{RO}, \mathbf{SO}, \mathbf{RQ}, \mathbf{SQ}),
\]
where:
\begin{enumerate}[itemsep=0pt,label=(\roman*)]
\item \(\mathbf{RO}\) (Real Object) is a stable idempotent with 
      significant noncommutativity to the ambient subalgebra, 
      modeling the “withdrawn” essence,
\item \(\mathbf{SO}\) (Sensual Object) is a projection or partial projector 
      in a subalgebra $\mathcal{B}\subseteq \mathcal{A}$, representing the vantage 
      or phenomenal domain; $\mathbf{SO}$ can partially overlap with $\mathbf{RO}$,
\item \(\mathbf{RQ}\) (Real Qualities) commute (fully or in part) with $\mathbf{RO}$ 
      but typically not with $\mathbf{SO}$, capturing latent or essential features 
      that are not exhausted in any single relational stance,
\item \(\mathbf{SQ}\) (Sensual Qualities) arise from or within $\mathbf{SO}$, 
      do not generally commute with $\mathbf{RO}$, and correspond to 
      the “relationally accessible” properties in a given vantage.
\end{enumerate}
\end{definition}

%=========================================================
% WITHDRAWAL AND PARTIAL OVERLAPS
%=========================================================

\section{Withdrawal and Partial Overlaps}
\label{sec:withdrawal}

\subsection{Commutators as “Withdrawal”}
One hallmark of OOO is that real objects withdraw from total relational capture. 
In UOR, we interpret this as the existence of non-zero commutators 
$[\mathbf{RO},X]\neq0$ for many $X \in \mathcal{A}$. 
If $\mathbf{RO}$ fully commuted with all $X$ in a relevant subalgebra, 
the real object would be fully “transparent” there. Instead, 
non-zero commutators signify “irreducibility” or “autonomy.”

\subsection{Overlap of \(\mathbf{RO}\) and \(\mathbf{SO}\)}
Partial overlap can be signaled by $\mathbf{RO}\,\mathbf{SO}\neq0$, 
so that $\mathbf{SO}$ “contacts” the real object in some dimension. 
If $\mathbf{RO}\,\mathbf{SO}=0$, the vantage sees none of the object’s essence. 
Thus the measure $\mathrm{Tr}(\mathbf{RO}\,\mathbf{SO})$ or 
$\|\mathbf{RO}\,\mathbf{SO}\|$ can quantify how strongly the real object 
is “immanent” in that phenomenal domain.

%=========================================================
% EXAMPLES AND GAUGE
%=========================================================

\section{Examples and Gauge Symmetries}
\label{sec:gauge}

\subsection{Matrix Example}
\begin{example}[Orthogonal or Overlapping Idempotents]
In a $4\times4$ matrix algebra, we can define:
\[
\mathbf{RO} = \mathrm{diag}(1,1,0,0),\quad
\mathbf{SO} = \mathrm{diag}(0,0,1,1).
\]
They are orthogonal: $\mathbf{RO}\,\mathbf{SO}=0$. 
Hence no “sensual qualities” appear from the vantage associated with $\mathbf{SO}$. 
Alternatively, if $\mathbf{SO} = \mathrm{diag}(1,0,1,0)$, partial overlap arises:
\[
\mathbf{RO}\,\mathbf{SO} 
= 
\mathrm{diag}(1,0,0,0),
\]
yielding a nonzero sub-block that can be interpreted as the “visible portion” 
of the real object in that vantage.
\end{example}

\subsection{Gauge Symmetries}
If $\mathcal{A}$ also includes gauge factors (e.g.\ $\mathrm{U}(1)\times \mathrm{SU}(2)\times \mathrm{SU}(3)$), 
one can interpret $\mathbf{RQ}$ as gauge-invariant blocks (mass, topological numbers, etc.) 
and $\mathbf{SQ}$ as gauge-transformed states that vary with vantage or internal symmetry. 
Hence the object can appear differently in different frames while retaining 
a stable “real core.”

%=========================================================
% SUPPLEMENT
%=========================================================

\section{Supplement: Further Clarifications}
\label{sec:supplement}

\subsection{Base-\texorpdfstring{\(b\)}{b} Expansions and Overlap}
Choosing a larger base $b$ typically yields finer partitions of $\mathcal{A}$ 
into digit operators $P_{d,k}$. This can \emph{increase} potential overlap 
between $\mathbf{RO}$ and $\mathbf{SO}$, since the vantage domain sees 
more granular details. A smaller $b$ lumps degrees of freedom, 
often resulting in greater orthogonality or “withdrawal.” 
Conceptually, $b$ might correlate with the “resolution” or “variety of categories” 
in a vantage.

\subsection{Explicit Commutator Norms Illustrating Withdrawal}
Consider 
\[
\mathbf{RO} 
= 
\begin{pmatrix}
1 & 0 & 0 & 0\\
0 & 1 & 0 & 0\\
0 & 0 & 0 & 0\\
0 & 0 & 0 & 0
\end{pmatrix},
\quad
X 
=
\begin{pmatrix}
0 & 1 & 0 & 0\\
1 & 0 & 0 & 0\\
0 & 0 & 0 & 1\\
0 & 0 & 1 & 0
\end{pmatrix}.
\]
Their commutator $[\mathbf{RO},X]$ has nonzero off-diagonal blocks. 
The operator norm of $[\mathbf{RO},X]$ can be computed 
(e.g.\ largest singular value). The result measures how strongly 
$\mathbf{RO}$ remains outside full assimilation by $X$. 
In OOO terms, it indicates how the real object “resists” translation 
into that operator’s vantage or relational domain.

\subsection{Gauge Symmetries Structuring \texorpdfstring{\(\mathbf{RQ}\)}{RQ} and \texorpdfstring{\(\mathbf{SQ}\)}{SQ}
}
If we embed a gauge group in $\mathcal{A}$, “real qualities” might be gauge invariants 
that do not transform, while “sensual qualities” vary under transformations 
(e.g.\ color charge, flavor, or representation labels). This duality parallels 
OOO’s claim that certain “inherent” aspects remain hidden, 
while “phenomenal” aspects shift with vantage.

%=========================================================
% CONCLUSION
%=========================================================

\section{Conclusion and Future Directions}

We have presented a unified explanation of how the \emph{Object-Oriented Ontology} 
quadruple object (real object, sensual object, real qualities, sensual qualities) 
can be concretely formalized in the \emph{Universal Object Reference} framework. 
Key points include:
\begin{itemize}[itemsep=0pt]
\item \textbf{Stable Idempotents and Noncommutativity:} 
  capturing “withdrawal” as commutator-based irreducibility,
\item \textbf{Base-$b$ Expansions:} 
  controlling how “coarse” or “fine” the vantage or relational domain is, 
  thereby modulating overlap between real and sensual poles,
\item \textbf{Gauge Symmetries:} 
  clarifying how real qualities remain stable under transformations, 
  while sensual qualities shift with vantage.

\end{itemize}
By illustrating these principles with matrix examples and subalgebra distinctions, 
we unify philosophical claims of OOO with a rigorous algebraic architecture. 
This invites further applications, such as modeling how objects appear in 
physical theories, how “withdrawing” aspects relate to gauge-invariant properties, 
and whether base-$b$ expansions might correlate with hierarchical observational 
or conceptual frameworks in metaphysics and epistemology.

\bigskip

%=========================================================
% BIBLIOGRAPHY
%=========================================================

\begin{thebibliography}{99}

\bibitem{HarmanQuadruple}
G.~Harman, 
\emph{The Quadruple Object}, 
Zero Books, 2011.

\bibitem{BryantOOO}
L.~Bryant, 
\emph{The Democracy of Objects}, 
Open Humanities Press, 2011.

\bibitem{MortonOOO}
T.~Morton, G.~Harman, et al., 
\emph{Object-Oriented Ontology: A Symposium}, 
Phil. Compass \textbf{14} (2019).

\bibitem{LounestoClifford}
P.~Lounesto,
\emph{Clifford Algebras and Spinors},
2nd ed., Cambridge University Press, 2001.

\bibitem{Connes}
A.~Connes,
\emph{Noncommutative Geometry},
Academic Press, 1994.

\bibitem{UORPaper}
UOR Collaboration,
\emph{Universal Object Reference (UOR): Algebraic Embeddings of Definable Domains},
(preprint, 2025).

\end{thebibliography}

\end{document}
