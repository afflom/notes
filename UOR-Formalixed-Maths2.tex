\documentclass[12pt]{article}
\usepackage{amsmath, amssymb, amsthm}
\usepackage{hyperref}
\usepackage{geometry}
\geometry{a4paper, margin=1in}

\title{Mathematical Formalization of Universal Object Reference (UOR) -- Part II}
\author{}
\date{}

\begin{document}

\maketitle

\section{Clifford Bundle Construction}

\subsection{Local to Global Structure}
\begin{itemize}
    \item \textbf{Clifford Algebra Fibers:} Given a smooth manifold $M$ with a Riemannian metric $g$, for each point $x \in M$ we attach a fiber $\mathcal{C}_x = \mathrm{Cl}(T_x M, g_x)$, the Clifford algebra of the tangent space at $x$ with respect to the quadratic form $g_x$.  Recall that $\mathrm{Cl}(T_x M, g_x)$ is the associative algebra generated by $T_x M$ with the defining relation 
    \[
       v \cdot w + w \cdot v = 2\,g_x(v,w)\,1,
    \] 
    for all $v,w\in T_x M$ (here $1$ denotes the multiplicative identity). Thus each fiber $\mathcal{C}_x$ is a finite-dimensional real associative algebra (of dimension $2^{\dim M}$) containing $T_x M$ (identified with the degree-1 elements) and encoding the metric structure at $x$. We then construct the disjoint union 
    \[
       \mathcal{C} = \bigsqcup_{x \in M} \mathcal{C}_x,
    \] 
    and endow it with the structure of a smooth fiber bundle $\pi: \mathcal{C} \to M$ by using local trivializations as follows.

    \item \textbf{Bundle Trivializations and Transition Functions:} Over a coordinate chart $(U, \varphi)$ on $M$, one can choose a local orthonormal frame of tangent vectors $e_1,\dots,e_n$ on $U$ (possible at least on sufficiently small neighborhoods, assuming $M$ is orientable and spin$^c$ so that a global Clifford bundle exists). This frame yields a local trivialization of the Clifford bundle over $U$ by identifying $\mathcal{C}_x$ with the model algebra $\Cl(\mathbb{R}^n, \eta)$ (where $\eta$ is the standard Euclidean metric) via the map sending each basis vector $e_i(x)\in T_x M$ to the corresponding basis element $E_i$ in $\Cl(\mathbb{R}^n,\eta)$. In overlapping charts, if $\{e_i\}$ and $\{e_i'\}$ are two orthonormal frames on $U \cap U'$, the transition function on $U \cap U'$ is given by the unique algebra isomorphism $\,\Psi: \Cl(\mathbb{R}^n,\eta) \to \Cl(\mathbb{R}^n,\eta)\,$ that sends the basis $E_i$ (from frame $\{e_i\}$) to $E'_i$ (from frame $\{e'_i\}$). Equivalently, if $O_{jj'}$ is the orthogonal change-of-basis matrix sending $e_j$ to $e'_{j'}$, then $\Psi$ acts on generators by $E_j \mapsto \sum_{j'} O_{jj'}\, E'_{j'}$, extending multiplicatively to all of $\Cl(\mathbb{R}^n)$. These transition maps preserve the algebraic relations (they are orthogonal transformations on the generators, hence send $v\cdot v$ to $v'\cdot v'$ appropriately), which ensures that on overlaps the local trivializations are compatible. In this way $\mathcal{C}$ becomes a well-defined smooth fiber bundle over $M$ whose fiber at $x$ is $\Cl(T_x M, g_x)$. The structure group of this bundle is (locally) $O(n)$ (or the spin group in a spin structure), acting via algebra automorphisms on the model fiber.

    \item \textbf{Diffeomorphism Consistency:} Any smooth diffeomorphism $\Phi: M \to M$ (especially those preserving the metric $g$) induces a natural map on the Clifford bundle, ensuring the bundle structure respects the manifold's symmetry. In particular, if $f: M \to M$ is a diffeomorphism and $df_x: T_x M \to T_{f(x)}M$ its differential at $x$, we obtain an induced algebra isomorphism $\tilde{f}_x: \Cl(T_x M, g_x) \to \Cl(T_{f(x)}M, g_{f(x)})$ defined by sending $v \in T_x M$ to $df_x(v) \in T_{f(x)}M$ and extending to all of $\Cl(T_x M)$ multiplicatively. If $f$ is an isometry (so that $g_{f(x)}(df_x(v), df_x(w)) = g_x(v,w)$), then $\tilde{f}_x$ preserves the Clifford multiplication structure exactly. More generally, we require that sections of the Clifford bundle transform covariantly under such base diffeomorphisms by the induced algebra maps. This means that if $s$ is a (local) section $s: U \to \mathcal{C}$ and $f: U \to f(U)$ is a diffeomorphism, then the pushed-forward section $f_* s$ is defined by $(f_* s)(y) := \tilde{f}_{f^{-1}(y)}(s(f^{-1}(y)))$ for $y \in f(U)$. In this way, changes of coordinates or symmetries of $M$ consistently lift to automorphisms of the Clifford bundle, ensuring the local algebraic data $\mathcal{C}_x$ is coherent under moving to a new reference frame.
\end{itemize}

\subsection{Algebraic Properties of the Clifford Bundle}
\begin{itemize}
    \item \textbf{Associative Algebra of Sections:} The space of global sections $\Gamma(\mathcal{C})$ of the Clifford bundle inherits a natural associative algebra structure defined pointwise. Given two sections $s, t \in \Gamma(\mathcal{C})$, their product section $u = s \cdot t$ is defined by $(s \cdot t)(x) := s(x)\,t(x)$, using the Clifford multiplication in the fiber $\mathcal{C}_x$. This product is associative because each fiber algebra $\mathcal{C}_x$ is associative, and it is smooth provided $s$ and $t$ are smooth sections. There is a natural identity element in this algebra of sections: namely the constant section $1(x) \equiv 1_{x}$ (where $1_x$ is the multiplicative identity in $\Cl(T_xM)$ for each $x$). Similarly, each fiber’s inversion (when it exists, e.g. for invertible elements) yields a notion of pointwise inverse for sections (though the set of invertible sections may be restricted). In summary, $\Gamma(\mathcal{C})$ forms an associative algebra (in fact, a sheaf of such algebras over $M$), which unifies all local Clifford algebras into a single global algebraic structure.

    \item \textbf{Differentiation and Exterior Derivative:} Since the Clifford bundle is a geometric (tensor-algebra-like) construction, it admits natural differentiation operators extending those on functions and differential forms. In particular, we can define an \emph{exterior derivative} $d$ on sections of $\mathcal{C}$ that extends the usual exterior derivative on differential forms. The idea is that each homogeneous component of a Clifford-algebra section behaves like a differential form of a certain degree. More precisely, recall that there is a canonical vector space isomorphism between the Clifford algebra $\Cl(T_xM)$ and the full exterior algebra $\Lambda^*(T_x^*M)$ (the identification is given by mapping a wedge of basis covectors to the corresponding “blade” in the Clifford algebra). Using this, any section $s \in \Gamma(\mathcal{C})$ can locally be written as a sum $s = \sum_{k=0}^n s^{(k)}$, where $s^{(k)}$ is an element of degree $k$ (an image of a $k$-form). We define $d s$ by acting as the usual exterior derivative on each $s^{(k)}$ component.  Thus $d(\sum_k s^{(k)}) := \sum_k d\,s^{(k)}$, where on the right-hand side $d$ is the standard exterior derivative (taking a $k$-form to a $(k+1)$-form). In local coordinates, if $s(x) = \frac{1}{k!} s_{i_1\ldots i_k}(x)\,e_{i_1}(x)\cdots e_{i_k}(x)$ (with each term an antisymmetric $k$-blade), then 
    \[
       d s(x) = \frac{1}{k!} \frac{\partial s_{i_1\ldots i_k}}{\partial x^j}(x)\,dx^j \wedge (e_{i_1}(x)\cdots e_{i_k}(x)) ,
    \] 
    which corresponds to a $(k+1)$-blade component of the Clifford algebra at $x$. In this way $d: \Gamma(\mathcal{C}) \to \Gamma(\mathcal{C})$ increases the grade by 1 and coincides with the usual de~Rham differential on pure differential form components. By construction $d^2=0$ (since it acts like the de~Rham differential on each part), and $d$ extends the familiar notions of gradient, curl, and divergence to the unified algebra of Clifford sections. (For example, if $f$ is a scalar field (grade 0 section), $df$ is the usual 1-form (grade 1) given by the gradient; if $V$ is a vector field (grade 1 section, identified with a 1-form via $g$), then $dV$ has a grade 2 part corresponding to the curl of $V$, etc.) In a more advanced treatment, one can also introduce a covariant derivative $\nabla$ on $\mathcal{C}$ (induced from the Levi-Civita connection on $M$) that acts as a derivation compatible with the Clifford product, but for our purposes the exterior derivative $d$ suffices to encode the idea of differentiating Clifford-valued data on $M$. 

    \item \textbf{Grading Structure:} Each Clifford fiber $\mathcal{C}_x$ carries a natural $\mathbb{Z}$-grading as a vector space (direct sum decomposition by grade). We have 
    \[
       \mathcal{C}_x = \bigoplus_{k=0}^{n} \mathcal{C}_x^{(k)}\,,
    \] 
    where $\mathcal{C}_x^{(k)}$ consists of all elements of $\mathcal{C}_x$ that are $k$-fold products of tangent vectors (and are linearly independent of lower-grade products). For example, $\mathcal{C}_x^{(0)}$ are scalars (just the real numbers times the identity $1_x$), $\mathcal{C}_x^{(1)}$ is identified with $T_x M$ (the vector elements), $\mathcal{C}_x^{(2)}$ consists of bivectors (formal oriented area elements $v\wedge w$), and so on up to $\mathcal{C}_x^{(n)}$ which is spanned by the pseudoscalar $e_1\wedge e_2\wedge \cdots \wedge e_n$ (an oriented volume element at $x$). This grading is compatible with the bundle structure, giving a decomposition of the entire Clifford bundle $\mathcal{C} = \bigoplus_{k} \mathcal{C}^{(k)}$ where $\mathcal{C}^{(k)} = \bigsqcup_{x} \mathcal{C}_x^{(k)}$ is the sub-bundle of grade-$k$ elements. We note that the Clifford product does \emph{not} preserve the grading strictly (for instance, multiplying a vector and a bivector yields a scalar plus a trivector in general), but it does satisfy a parity property: the product of a grade-$i$ and grade-$j$ element lies in grades $i+j-2m$ (for various $m$ depending on how many metric contractions occur). In particular, the even-graded part $\mathcal{C}_x^{\text{even}} = \bigoplus_{k \text{ even}} \mathcal{C}_x^{(k)}$ is a subalgebra (closed under multiplication), and similarly for the odd part. The grading allows one to project any element onto its $k$-vector components. In summary, each fiber is a graded algebra capturing scalar, vector, bivector, ..., pseudoscalar components in one structure, and this grading extends over $M$. This will be crucial for defining \emph{universal coordinates} later, where an object’s coordinates consist of pieces in each grade.
\end{itemize}

\section{Lie Group Symmetry and Automorphisms}

\subsection{Group Action on $M$ and Fibers}
\begin{itemize}
    \item \textbf{Lifted Action of a Lie Group:} Let $G$ be a Lie group that acts smoothly on $M$ by isometries (i.e. diffeomorphisms that preserve the metric $g$). We denote the action as $G \times M \to M$, $(g, x) \mapsto g\cdot x$, where $g\cdot x$ is the image of $x$ under $g$. This action on the base manifold lifts naturally to an action on the Clifford bundle by bundle automorphisms. Specifically, for each $g \in G$, we have an induced map $\Phi(g): \mathcal{C} \to \mathcal{C}$ that covers the base action $g: M \to M$. The map $\Phi(g)$ on fibers is defined as an algebra isomorphism 
    \[
       \Phi(g)_x : \mathcal{C}_x \;\to\; \mathcal{C}_{\,g\cdot x}\,,
    \] 
    determined by the differential $dg_x: T_x M \to T_{g\cdot x}M$.  Concretely, if $a_x \in \Cl(T_xM)$, then $\Phi(g)_x(a_x)$ is the element of $\Cl(T_{g\cdot x}M)$ obtained by applying $dg_x$ to each vector factor in $a_x$. For example, if $a_x = v_1 v_2 \cdots v_k$ (the Clifford product of tangent vectors $v_i \in T_xM$), then 
    \[
       \Phi(g)_x(a_x) := (dg_x(v_1))\, (dg_x(v_2)) \cdots (dg_x(v_k)) \in \Cl(T_{g\cdot x}M)\,. 
    \] 
    This is well-defined and extends by linearity to all of $\mathcal{C}_x$, because $g$ preserves the metric (so it preserves the defining relation of the Clifford algebra: $dg_x(v)\,dg_x(w)+dg_x(w)\,dg_x(v) = 2\,g_{g\cdot x}(dg_x(v), dg_x(w))$ if $v w + w v = 2\,g_x(v,w)$). In this way, each $g\in G$ induces an algebra isomorphism between the Clifford fibers at $x$ and at $g\cdot x$. Collectively, $\Phi: G \to \operatorname{Aut}(\mathcal{C})$ (with $g \mapsto \Phi(g)$) defines a smooth \emph{bundle automorphism group action} on $\mathcal{C}$.

    \item \textbf{Action on Sections (Covariance):} The induced action on the space of sections $\Gamma(\mathcal{C})$ is given by a pullback or pushforward operation. Given $s \in \Gamma(\mathcal{C})$ (a section assigning to each $x$ an element $s(x) \in \mathcal{C}_x$), we define a new section $g \cdot s$ by 
    \[
        (g\cdot s)(x) := \Phi(g)_{\,g^{-1}\cdot x}\big(s(g^{-1}\cdot x)\big)\,.
    \] 
    This formula says: to evaluate $(g\cdot s)$ at $x$, take the original section $s$ at the point $g^{-1}\cdot x$ (which is the point that $x$ comes from under $g$), and then transport that element to the fiber at $x$ using $\Phi(g)$. This is the usual way a bundle automorphism induces an action on sections (it is essentially the pushforward of the section by $g$). One can easily check that this defines a left action of $G$ on $\Gamma(\mathcal{C})$ and that it respects the algebra structure: $(g\cdot s) \cdot (g \cdot t) = g \cdot (s \cdot t)$, meaning the Clifford multiplication is $G$-equivariant. In other words, the $G$-action commutes with taking products, so the symmetry is an \emph{automorphism} of the entire algebraic structure of sections, not just the underlying set.

    \item \textbf{Invariance of the Coherence Norm:} Because $G$ acts by isometries on $M$, the lifted action $\Phi(g)$ preserves the inner product structure of each fiber (to be defined precisely in Section~3). Intuitively, since $dg_x$ is orthogonal (metric-preserving), it carries an orthonormal frame at $x$ to an orthonormal frame at $g\cdot x$, and hence maps an orthonormal basis of $\mathcal{C}_x$ to an orthonormal basis of $\mathcal{C}_{g\cdot x}$. It follows that the canonical inner product $\langle\cdot,\cdot\rangle_c$ on the fibers (and the associated norm $|\cdot|_c$) is $G$-invariant. In particular, for any $a_x \in \mathcal{C}_x$ we have 
    \[
       |\Phi(g)_x(a_x)|_c \;=\; |a_x|_c\,,
    \] 
    since $\Phi(g)_x$ is an orthonormal linear isomorphism between the inner product spaces $(\mathcal{C}_x,\langle\cdot,\cdot\rangle_c)$ and $(\mathcal{C}_{g\cdot x},\langle\cdot,\cdot\rangle_c)$. Consequently, the norm of a section $s(x)$ is unchanged under the symmetry: $| (g\cdot s)(x) |_c = |s(g^{-1}\cdot x)|_c = |s(y)|_c$ if $y = g^{-1}\cdot x$. Thus, all “lengths” or magnitudes defined by the coherence norm are objective, i.e. independent of the particular reference frame or symmetry transformation. This invariance property is crucial: it ensures that any criterion based on the norm (such as a minimization principle or consistency check) will be equally valid in all coordinate systems related by $G$.
\end{itemize}

\subsection{Infinitesimal Generators}
\begin{itemize}
    \item \textbf{Lie Algebra of Derivations:} Let $\mathfrak{g}$ be the Lie algebra of the Lie group $G$. Each element $X \in \mathfrak{g}$ corresponds to an infinitesimal generator of the group action on $M$, i.e. a Killing vector field $X_M$ on $M$ (the velocity field of the one-parameter subgroup $\exp(tX)$ acting on $M$). This infinitesimal action lifts to the Clifford bundle as well. Differentiating the family of automorphisms $\Phi(\exp(tX))$ at $t=0$ yields a linear operator (a derivation) $D_X$ acting on sections of $\mathcal{C}$. Concretely, for a section $s \in \Gamma(\mathcal{C})$, the Lie derivative (infinitesimal action) along $X$ is 
    \[
      D_X(s) := \frac{d}{dt}\Big|_{t=0} \Big(\exp(tX) \cdot s\Big)\,,
    \] 
    where $\exp(tX)\cdot s$ is the pushed-forward section as defined above. The result $D_X(s)$ is again a section of $\mathcal{C}$. One can show that $D_X$ is a derivation of the Clifford-algebra structure, meaning it satisfies the Leibniz rule 
    \[
      D_X(s \cdot t) = (D_X s)\cdot t + s \cdot (D_X t)\,,
    \] 
    for any sections $s,t$. This holds because $\Phi(\exp(tX))$ are algebra automorphisms for all $t$, so differentiating the identity $\Phi(\exp(tX))(s\cdot t) = \Phi(\exp(tX))(s)\cdot \Phi(\exp(tX))(t)$ at $0$ gives the Leibniz rule for $D_X$. In intuitive terms, the Lie algebra element $X$ acts on local objects $a_x \in \mathcal{C}_x$ by the differential of the group action, and since the group action preserves the product structure, the generator preserves the algebraic structure to first order. The collection of all such infinitesimal generators $\{D_X: X\in \mathfrak{g}\}$ forms a Lie algebra of derivations acting on $\Gamma(\mathcal{C})$. Indeed, one can verify that $[D_X, D_Y] = D_{[X,Y]}$ (where the left side is the commutator of operators on sections and the right side is the derivation corresponding to the Lie bracket in $\mathfrak{g}$). Thus, the $G$-symmetry gives rise to a representation of the Lie algebra $\mathfrak{g}$ by derivations of the Clifford-algebra bundle.

    \item \textbf{Curvature Operator and Lie Derivative Algebra:} In the context of a global symmetry group action, the induced Lie derivative operators $D_X$ commute in the same way as their corresponding vector fields. Equivalently, as noted, $D_X D_Y - D_Y D_X = D_{[X,Y]}$. This means that the family $\{D_X\}$ is a homomorphism of Lie algebras (from $(\mathfrak{g}, [\cdot,\cdot])$ into the Lie algebra of derivations on $\Gamma(\mathcal{C})$). In differential geometric terms, this indicates there is \emph{no curvature associated with the $G$-action itself} — transporting an object around an infinitesimal loop generated by $X$ and $Y$ brings one back to the same result as doing it in the opposite order, up to the naturally expected difference given by $[X,Y]$. We can formalize this by introducing a curvature operator $R$ that measures the failure of two derivations to commute: define, for any $X,Y \in \mathfrak{g}$,
    \[
        R(X, Y) := D_X D_Y - D_Y D_X - D_{[X,Y]}\,.
    \] 
    In a perfectly flat (integrable) symmetry action, we have $R(X,Y) = 0$ identically. Indeed, for the derivations coming from a genuine Lie group action, $R(X,Y)$ vanishes as noted above. This situation is analogous to having a flat connection on the bundle with respect to the symmetry flows. By contrast, if one were to consider more general (not globally integrable) transformations on the bundle, a nonzero curvature operator would signal an inconsistency or path-dependence in how one could transport objects around loops. In our case, thanks to $G$ being a true symmetry group of $M$, the induced Lie derivatives form a flat (curvature-free) system. This reinforces the idea that the UOR framework encodes the principle of general covariance: truths or object properties are invariant under any sequence of symmetry transformations (there is no ambiguity or accumulation of error from moving between references in different orders). Furthermore, the derivations $D_X$ often coincide with familiar operations: for instance, if $X$ is a generator of rotations, $D_X$ acting on a section will correspond to the Lie derivative encoding how that section rotates; if $X$ generates time translations, $D_X$ corresponds to the time derivative (evolution) of the object, etc. The absence of curvature in this symmetry algebra means these operations satisfy the usual expected commutation relations (e.g., time-derivative commuting with spatial rotation up to the rotation of the time-derivative, and so on), consistent with the physical interpretation of a well-defined symmetry group acting on the system.
\end{itemize}

\section{Coherence and Universal Coordinates}

\subsection{Coherence Norm and Invariant Quantities}
\begin{itemize}
    \item \textbf{Canonical Inner Product on Fibers:} Each fiber $\mathcal{C}_x$ is equipped with a natural inner product $\langle\cdot,\cdot\rangle_c$ that turns it into an inner product space (in fact, a finite-dimensional Hilbert space over $\mathbb{R}$). This inner product is defined in a coordinate-invariant way by using the metric $g_x$ to declare an orthonormal basis of “blade” elements.  Concretely, choose an orthonormal basis $(e_1,\dots,e_n)$ of $T_x M$ (with respect to $g_x$), and consider the corresponding Clifford algebra basis consisting of all products $e_{i_1}e_{i_2}\cdots e_{i_k}$ with $i_1 < \cdots < i_k$ (over all $k=0,\dots,n$). This set forms a basis of $\mathcal{C}_x$. We then declare this basis to be orthonormal in $\mathcal{C}_x$ (with respect to $\langle\cdot,\cdot\rangle_c$) by stipulating 
    \[
       \Big\langle e_{i_1}\cdots e_{i_k}\,,\, e_{j_1}\cdots e_{j_\ell}\Big\rangle_c \;=\; 
       \begin{cases}
           1 & \text{if } k=\ell \text{ and } \{i_1,\dots,i_k\} = \{j_1,\dots,j_k\}\ (\text{as sets, in the same order}),\\[6pt]
           0 & \text{otherwise.}
       \end{cases}
    \] 
    In particular, the scalar $1$ (grade-0 element) and each orthonormal vector $e_i$ (grade-1) and each orthonormal $k$-blade $e_{i_1}\cdots e_{i_k}$ are all chosen to have unit length and be mutually orthogonal. This defines $\langle\cdot,\cdot\rangle_c$ on basis elements and extends bilinearly to all of $\mathcal{C}_x$. Because any two choices of orthonormal frame at $x$ are related by an orthogonal transformation, and such a transformation sends the blade basis to another blade basis with the same inner products (orthogonal transformations have determinant $\pm 1$ and just permute or flip signs of basis blades), this inner product is actually independent of the choice of frame. We call the norm induced by this inner product the \emph{coherence norm}: 
    \[
        |a_x|_c := \sqrt{\langle a_x,\,a_x\rangle_c}\,,
    \] 
    for $a_x \in \mathcal{C}_x$. By construction, for any $a_x = \sum_k a_x^{(k)}$ (decomposed into grades), the cross-terms between different grades vanish in the norm: $\langle a_x^{(i)}, a_x^{(j)}\rangle_c = 0$ if $i\neq j$. Thus $|a_x|_c^2 = \sum_k |a_x^{(k)}|_c^2$, meaning the norm is \emph{grade-orthogonal}. Each grade contributes independently (like orthogonal directions in a vector space) to the overall magnitude of $a_x$. This norm is positive-definite (in particular, $|a_x|_c = 0$ if and only if $a_x = 0$ in the algebra) and so each fiber becomes an inner product space. Finally, as noted earlier, this inner product is $G$-invariant: if $a_x \in \mathcal{C}_x$ and $b_x \in \mathcal{C}_x$, and $g\in G$, then $\langle \Phi(g)_x(a_x),\,\Phi(g)_x(b_x)\rangle_c = \langle a_x, b_x\rangle_c$. Therefore $|\cdot|_c$ provides a coordinate- and symmetry-invariant measure of the size (or “consistency”) of any object in the UOR framework.

    \item \textbf{Norm Minimization and Preferred Representation:} One important role of the coherence norm $|\cdot|_c$ is to provide a criterion for selecting the most \emph{coherent} or preferred representation of an object when multiple representations are possible. In many cases, an abstract object or concept can be realized in the Clifford bundle in more than one way or with additional superfluous components. The coherence norm gives a quantitative measure of any such redundancy or inconsistency. The general principle is that out of all equivalent representations of the same underlying object, the UOR framework singles out those that minimize the coherence norm as the most natural or consistent ones. For example, suppose an object $a_x \in \mathcal{C}_x$ can be written as $a_x = b_x + c_x$ in two different ways (perhaps $b_x$ and $c_x$ live in different subspaces or come from different coordinate constructions, but their sum represents the full object). If $b_x$ and $c_x$ are truly two facets of the same single object, one expects that they should combine cleanly without leaving leftover “disagreement.” In terms of the norm, this means that $b_x$ and $c_x$ should not point in orthogonal directions in the Clifford space (otherwise $|a_x|_c^2 = |b_x|_c^2 + |c_x|_c^2$ would be strictly larger than either part alone). The coherence principle posits that the object should be represented in a way that avoids such orthogonal decomposition unless required by the object’s nature. In practice, the condition for a preferred representation is often that any decomposition of an object into multiple representational parts should lie along the same direction in $\mathcal{C}_x$ (i.e. one part contains the other, rather than being orthogonal). When this is achieved, the norm of the sum $a_x$ is as small as possible (in fact equal to the sum of norms in the fully coherent case, or simply the norm of one part if the other is redundant). Thus, minimizing $|a_x|_c$ subject to representing the desired content tends to eliminate spurious components and enforce a canonical form. Another way to say this is that the inner product penalizes any mismatch or misalignment between different pieces of the representation, so the minimum-norm configuration occurs when all pieces perfectly reinforce each other, yielding a unified representation.

    \item \textbf{Enforcing Consistency Across Representations:} The coherence norm is especially powerful in situations where an object might be described in multiple ways (multiple coordinate charts, multiple bases, or multiple levels of abstraction). In the UOR framework, such an object would be simultaneously embedded via each representation, potentially yielding multiple components in $\mathcal{C}_x$ that ideally describe the same entity. We demand that these components align to actually form a single object, and the coherence norm provides a way to enforce this alignment. If $a_x^{(1)}, a_x^{(2)}, \dots, a_x^{(r)}$ are $r$ different representations (in possibly different subspaces or grades) of what should conceptually be the same object, then consistency requires that they sum to a total object $a_x = a_x^{(1)} + \cdots + a_x^{(r)}$ whose norm is not larger than necessary. In fact, in a perfect scenario $a_x$ should lie entirely in the intersection of those representational subspaces, or equivalently there should exist a single element common to all representations. If there are discrepancies, those will appear as additional orthogonal components in the sum, causing $|a_x|_c$ to increase. By striving to minimize $|a_x|_c$, the framework ensures that the only contributions to $a_x$ are those that are universally agreed upon by all representations. In formal terms, one might set up an optimization: choose $a_x^{(1)},\ldots,a_x^{(r)}$ (varying within their representational domains) such that some constraint (like matching certain observed values in each representation) holds, and minimize $|a_x^{(1)}+\cdots+a_x^{(r)}|_c$. The minimum will occur when all $a_x^{(i)}$ are equal (or as equal as the constraints allow), thereby enforcing $a_x^{(1)} = a_x^{(2)} = \cdots = a_x^{(r)}$ in the ideal case. When this happens, the object is \emph{fully coherent}: it has a single representation that suffices for all purposes, and the norm simply reflects the intrinsic size of the object rather than any representational conflict. This principle underlies how UOR treats, for example, an equation expressed in different coordinate systems or a number expressed in different bases: only when the coordinate expressions or base expansions actually denote the same mathematical object will the coherence norm be minimized (indeed, zero additional norm from differences), signaling a consistent truth across those representations.
\end{itemize}

\subsection{Base Decomposition and Universal Embedding}
\begin{itemize}
    \item \textbf{Grade Decomposition and Universal Coordinates:} As noted, any element $a_x \in \mathcal{C}_x$ can be uniquely decomposed into its homogeneous grade components: 
    \[
       a_x = a_x^{(0)} + a_x^{(1)} + \cdots + a_x^{(n)}\,,
    \] 
    where $a_x^{(k)} \in \mathcal{C}_x^{(k)}$. If we further choose a local orthonormal basis $(e_1,\dots,e_n)$ of $T_xM$, we can express each grade component in coordinates. For instance, writing $E_{i_1\cdots i_k} := e_{i_1}e_{i_2}\cdots e_{i_k}$ for the basis $k$-blade corresponding to indices $i_1 < \cdots < i_k$, we can expand 
    \[
       a_x^{(k)} = \sum_{i_1<\cdots<i_k} A_{i_1\cdots i_k}(x)\; E_{i_1\cdots i_k}\,,
    \] 
    for some real coefficients $A_{i_1\cdots i_k}(x)$. These coefficients $A_{i_1\cdots i_k}(x)$ serve as the \emph{universal coordinates} of the object $a_x$ with respect to the chosen frame. Collecting all grades, $a_x$ is described by the collection of all its coordinates $\{A_{i_1\cdots i_k}(x)\}_{k=0}^n$. This is a very detailed coordinate system that includes not only the usual coordinates (for the vector part, etc.) but also coordinates for every higher-grade component (bivector, trivector, etc.). In principle, this coordinate description is universal in the sense that it can describe \emph{any} kind of object: a pure scalar will have only $A(x)$ in grade 0 and zeros elsewhere; a vector will have components only in grade 1; a plane or bivector quantity (like an oriented area element or a complex number interpreted as a 2D rotation) will have components only in grade 2; and a general object can have a combination. The coherence norm, as described above, is precisely the Euclidean norm on this full coordinate vector $(\{A_{i_1\cdots i_k}(x)\})$. The requirement that an object’s multiple representations agree can be rephrased as saying that in such a full coordinate list, certain entries (that represent the same conceptual content in different grades or modes) must coincide or one set must be zero, etc. Thus, the grade/base decomposition of $a_x$ provides a systematic way to embed all information about an object into a single coordinate tuple, making it feasible to compare and unify disparate representations. We call this exhaustive expression of an object across all grades its \textbf{universal coordinate representation}.

    \item \textbf{Universal Embedding Theorem:} \emph{Any mathematical object or structure can be embedded into the UOR framework as a Clifford bundle element (or collection of elements) in a natural way.}  In less formal terms, UOR is universal in its representational capacity: given any object from a mathematical domain (number, vector, tensor, algebraic structure, logical proposition, etc.), one can find an equivalent realization of that object as an element $a_x \in \mathcal{C}_x$ for some $x \in M$, possibly using a combination of grades to encode the necessary information. Because the Clifford algebra contains (as subspaces) copies of many familiar algebraic systems (scalars, oriented multi-vectors, spinors, matrices, etc.), it is flexible enough to serve as a host for virtually any kind of data. We outline the idea of the proof/justification:
    \begin{itemize}
       \item \textit{(Existence of an encoding)}: Start with an arbitrary mathematical object $\mathcal{O}$. We must construct a corresponding UOR object $(x, a_x)$, with $a_x \in \Cl(T_xM)$, such that the content and relations of $\mathcal{O}$ are preserved. If $\mathcal{O}$ is purely numerical (e.g. an integer or real number), one can simply take $a_x$ to be that number times the identity $1_x$ in the Clifford algebra (see the next section for examples like natural numbers). If $\mathcal{O}$ is a vector in some vector space $V$, and $\dim V \le n = \dim M$, we can embed $V$ into $T_xM$ (via a linear monomorphism) and let $a_x$ be the corresponding Clifford vector. More generally, for a tuple or list of numbers, one can assign each entry to a component of $a_x$ along a different basis blade. For a structured object like a matrix or a polynomial, one can use a combination of grades or even multiple points in $M$ to encode the two-dimensional array structure or sequence structure. In theory, since the Clifford algebra has dimension $2^n$, any finite amount of discrete data can be encoded in a single $\Cl(T_xM)$ as long as $2^n$ exceeds the “size” of the data representation needed. For infinite or more abstract objects (like an infinite sequence or a formal system), one can consider either an appropriate limit or an extended version of the framework (for instance, using an infinite-dimensional manifold or a projective limit of Clifford algebras), but such considerations are beyond the scope of this finite-dimensional formalism.
       \item \textit{(Preservation of structure)}: It is not enough to embed objects as raw data; we also want that the natural operations or relations involving $\mathcal{O}$ correspond to natural operations in the UOR setting. The Clifford algebra’s operations (addition, multiplication, and even the Lie algebra of bivectors under commutator) are rich enough to emulate many structures. For example, logical propositions might be embedded as idempotent elements of $\mathcal{C}_x$ (where conjunction is modeled by multiplication), or group elements might be represented by invertible elements of $\mathcal{C}_x$ (with group operation as multiplication). The \textit{universal embedding theorem} asserts that for any structure, we can find a representation such that these correspondences hold. In practice, one constructs homomorphisms from the structure’s axiomatic operations to those available in UOR. If the object is an element of a group, we ensure its image $a_x$ in $\mathcal{C}_x$ satisfies the same relations (e.g. $a_x \cdot b_x$ corresponds to group multiplication of the original elements, etc.). If the object is an element of a field or ring, one can use the scalar subalgebra (or another subalgebra isomorphic to the field) to embed it so that addition and multiplication agree. If the object is a geometric entity (say a line or plane in some space), one can use a suitable bivector or trivector in $\Cl(T_xM)$ to represent it, since bivectors represent oriented planes, etc.
       \item \textit{(Uniqueness and universality)}: While an embedding as described might not be unique (indeed, there could be many ways to encode a given object in $\mathcal{C}_x$), the UOR philosophy is that there is a “canonical” or most coherent choice which will be picked out by the coherence norm minimization and symmetry considerations. The theorem is called universal because it implies no essential loss of generality by working in the UOR framework: one can always translate any mathematical scenario into UOR terms. In categorical language, one might say UOR is a universal domain or a *universe* in which all mathematical objects live as concrete elements (somewhat analogous to set theory being a universal foundation for mathematics, but here with a geometric-algebraic flavor).
    \end{itemize}
    A full formal proof of this theorem would require a construction for each type of mathematical structure and verifying the embedding preserves all required properties, which is beyond the scope of this exposition. However, the examples in the next section illustrate the concept for several important cases, giving evidence of the breadth of structures that fit into UOR.
\end{itemize}

\section{Illustrative Examples and Special Cases}

\subsection{Euclidean and Relativistic Cases}
\begin{itemize}
    \item \textbf{Euclidean Space $\mathbb{R}^n$ and $SO(n)$ Symmetry:} As a simple but fundamental example, take $M = \mathbb{R}^n$ with the standard Euclidean metric $g_{ij} = \delta_{ij}$. The tangent space at each point is just $\mathbb{R}^n$ itself, and the Clifford algebra $\Cl(T_x M, g_x)$ is isomorphic to $\Cl(\mathbb{R}^n,\text{std})$, which we denote $\Cl(n,0)$ (an $n$-dimensional Euclidean Clifford algebra). Because $\mathbb{R}^n$ is contractible and has a global orthonormal frame (the constant unit vectors $e_1,\dots,e_n$), the Clifford bundle is trivial: $\mathcal{C} \cong M \times \Cl(n,0)$. A smooth section of $\mathcal{C}$ is just a smooth function $\mathbb{R}^n \to \Cl(n,0)$. The global section algebra is therefore isomorphic to the space of all functions on $\mathbb{R}^n$ valued in $\Cl(n,0)$. The group $G = SO(n)$ acts by rotations on the base $M = \mathbb{R}^n$ (one may also include translations to consider the full Euclidean group, but for simplicity consider $SO(n)$ through the origin). An element $R \in SO(n)$ sends a point $x \in \mathbb{R}^n$ to $R x$, and the induced action $\Phi(R)$ on the bundle sends a fiber element $a_x \in \Cl(n,0)$ to $a_{R x}' \in \Cl(n,0)$ by applying the orthonormal linear map $R$ to each vector factor in $a_x$. This recovers the familiar notion from geometric algebra: $SO(n)$ is generated by bivectors (grade-2 elements) in the Clifford algebra, and acting by a rotation on a multivector can be represented internally via conjugation by certain Clifford elements (spinors). In the UOR view, we treat the rotation simply as an automorphism $\Phi(R)$ that moves the point and correspondingly transforms the object’s coordinates in each fiber. The coherence norm in this case is just the standard Euclidean norm on multivectors. For example, if $a_x = (a_0 + a_i e_i + \frac{1}{2}a_{ij} e_i e_j + \cdots)$ at some point (with coordinates $a_0, a_i, a_{ij}, \dots$), then $|a_x|_c^2 = a_0^2 + \sum_i a_i^2 + \sum_{i<j} a_{ij}^2 + \cdots$. This norm is clearly invariant under rotations of the coordinates. As a concrete illustration, consider a vector $v_x = v_i e_i$ at point $x$. Under a rotation $R$, $v_x$ moves to $(R\cdot v)_{R x} = v_i (R e_i)$ at the rotated point, and the norm $|v_x|_c^2 = \sum_i v_i^2$ remains the same because $R$ being orthogonal implies $\sum_i v_i^2 = \sum_i (v'_i)^2$ for the rotated components $v'_i$. Thus, the UOR framework in this case reproduces ordinary Euclidean geometry and multivector algebra, showing that it is a true generalization of classical coordinate geometry. All the familiar concepts (dot product, wedge product, rotations, etc.) are contained as special cases.

    \item \textbf{Minkowski Space and Relativistic Physics:} Next, consider a pseudo-Riemannian example: let $M$ be four-dimensional Minkowski spacetime (e.g. $M \cong \mathbb{R}^4$ with metric signature $(+,-,-,-)$). Each tangent space $T_x M$ is isomorphic to the Minkowski vector space $\mathbb{R}^{3,1}$, and the Clifford algebra $\Cl(T_xM, g_x)$ is isomorphic to $\Cl(3,1)$, the Clifford algebra of a 4D space with one timelike and three spacelike dimensions. This algebra $\Cl(3,1)$ is well-known in physics as the Dirac algebra (isomorphic to the $4\times4$ real matrices that can represent gamma matrices, or equivalently the algebra generated by the Dirac gamma matrices $\gamma^\mu$ with $\gamma^\mu \gamma^\nu + \gamma^\nu \gamma^\mu = 2\eta^{\mu\nu}$). The bundle $\mathcal{C}$ in this case is again trivial (Minkowski space is also contractible and admits a constant frame), so we essentially have a single copy of $\Cl(3,1)$ at each point. The symmetry group $G$ here can be taken as the proper Lorentz group $SO(3,1)^+$ (or the full Poincaré group including translations, if we allow base translations as symmetries). The Lorentz group acts on $M$ by the usual linear transformations on coordinates (preserving the Minkowski metric), and this lifts to automorphisms of the Clifford bundle. In practice, a Lorentz transformation $\Lambda$ sends a tangent vector $v^\mu \partial_\mu$ at $x$ to $(\Lambda v)^\mu \partial_\mu$ at $\Lambda x$, and correspondingly $\Phi(\Lambda)_x$ sends the Clifford element $a_x$ (expressed in some frame at $x$) to the element $a_{\Lambda x}'$ which is obtained by replacing each $\gamma^\mu$ in the product representation of $a_x$ with $\Lambda^\mu{}_{\nu}\, \gamma^\nu$ (the rotated combination at the new point). The coherence norm is defined just as before: one chooses an orthonormal frame of basis vectors $(e_0,e_1,e_2,e_3)$ with $e_0$ timelike unit, $e_i$ spacelike units, then all basis blades $e_{i_1}\cdots e_{i_k}$ are orthonormal in the Clifford sense. In this case the norm $|a_x|_c^2 = \langle a_x, a_x\rangle_c$ will not be positive-definite in the same way if one used the Minkowski metric to define it; however, our definition of $\langle\cdot,\cdot\rangle_c$ treats all basis blades as orthonormal formal vectors (essentially using the Euclideanized version of the algebra for norm purposes). This means $|\cdot|_c$ remains $SO(3,1)$-invariant (since Lorentz transformations also just permute/boost the blade components) but yields a positive magnitude for any nonzero $a_x$. With this setup, one can represent relativistic physical quantities in UOR. For instance, a charged particle’s current density $J^\mu(x)$ (a vector field on spacetime) can be seen as a grade-1 section of $\mathcal{C}$; the electromagnetic field $F_{\mu\nu}(x)$ (an antisymmetric tensor) can be represented as a grade-2 section (a bivector field) in the Clifford bundle; and the spacetime volume element $\epsilon_{\mu\nu\rho\sigma}$ is essentially the grade-4 pseudoscalar (which can act as the unit oriented volume in the algebra). The field equations of electromagnetism or other relativistic theories can then be elegantly written using Clifford algebra operations (this is the idea of the “spacetime algebra” formulation of physics). The UOR framework doesn’t change the physics in this case; it simply provides a single unified algebraic structure at each event $x$ where scalars, 4-vectors, bivectors (like $E\wedge B$ fields), and so on, all live together and can be directly added or multiplied. Coherence in this context might relate to ensuring that a physical law written in different reference frames or gauge choices yields the same $a_x$ up to the symmetry transformation. For example, an electromagnetic field can be described by electric and magnetic components $(\mathbf{E}, \mathbf{B})$ which depend on the observer’s frame; UOR would embed the field in a frame-independent way as a single bivector $F_x$, and coherence would ensure that if one computes $F_x$ from $(\mathbf{E},\mathbf{B})$ in one frame or from $(\mathbf{E}',\mathbf{B}')$ in another frame, one gets the same bivector $F_x$ (the difference between the two descriptions is just a Lorentz transform, part of $G$). Thus UOR recapitulates the invariance of physical laws under Lorentz transformations in a single-object formalism.

\end{itemize}

\subsection{Number Theory and Symbolic Embeddings}
\begin{itemize}
    \item \textbf{Natural Numbers in Multiple Bases:} To demonstrate the ability of UOR to handle multiple representations of the same object, consider the embedding of a natural number with its expansions in different bases. Let $N$ be a natural number (e.g. $N=42$). We want to embed $N$ into the Clifford bundle in such a way that, for instance, its decimal (base-10) representation and its binary (base-2) representation are both realized and enforced to represent the same quantity. Choose an arbitrary reference point $x \in M$ and work in the fiber $\mathcal{C}_x$. We can designate two sets of orthogonal basis vectors in $T_xM$ (and corresponding Clifford basis elements) to play the role of “place-value axes” for base-10 and base-2 respectively. For example, pick one unit vector $u \in T_xM$ and assign it the property that $u^2 = 10\,1_x$ (so $u$ acts like a root of the scalar 10 in the algebra), and similarly pick another vector $v \in T_xM$ such that $v^2 = 2\,1_x$. (These choices can be done in a larger-dimensional tangent space, since we might need to accommodate that $u$ and $v$ should commute; one simple way is to ensure $u$ and $v$ anticommute as Clifford generators but consider powers of them separately. However, for conceptual clarity, assume we can treat the $u$ and $v$ directions independently for encoding purposes.) Now represent $N$ as follows:
    \[
        a_x = \underbrace{d_k d_{k-1} \cdots d_0}_{\text{(base-10 digits of $N$)}} \quad \text{interpreted as} \quad a_x^{(10)} = d_0\,1_x + d_1\,u + d_2\,u^2 + \cdots + d_k\,u^k,
    \] 
    and 
    \[
        a_x = \underbrace{b_m b_{m-1} \cdots b_0}_{\text{(binary digits of $N$)}} \quad \text{interpreted as} \quad a_x^{(2)} = b_0\,1_x + b_1\,v + b_2\,v^2 + \cdots + b_m\,v^m,
    \] 
    where $d_i \in \{0,\dots,9\}$ are the decimal digits of $N$ (so $N = \sum_{i=0}^k d_i 10^i$) and $b_j \in \{0,1\}$ are the binary bits of $N$ ($N=\sum_{j=0}^m b_j 2^j$). Here we are treating powers of $u$ and powers of $v$ as representing successive place values in base-10 and base-2, respectively. The key idea is that we embed both expansions into the \emph{same} Clifford element $a_x$. In a purely algebraic sense, $a_x^{(10)}$ and $a_x^{(2)}$ might look like different expressions, but for them to represent the same $a_x$ we require that 
    \[
        a_x^{(10)} = a_x^{(2)} \in \mathcal{C}_x\,.
    \] 
    This equation enforces the equality $\sum_i d_i u^i = \sum_j b_j v^j$ within the Clifford algebra. Because $u$ and $v$ correspond to different base directions, this equality will hold in $\Cl(T_xM)$ if and only if the numeric values match when evaluated (since $u$ and $v$ are independent directions, the equality separates into matching the scalar part, the $u$-line, the $u^2$-plane, etc., with the corresponding $v$-parts). Essentially, the coefficients must satisfy the usual equality of the two expansions of $N$. The coherence norm now plays a role: if there were any discrepancy between the two representations (say they encoded different numbers or one had an extra part that the other didn’t), then $a_x^{(10)} - a_x^{(2)}$ would be a nonzero element contributing to $|a_x|_c$. In the fully coherent case, $a_x^{(10)}$ and $a_x^{(2)}$ are exactly the same element, so $a_x = a_x^{(10)} = a_x^{(2)}$ and there is no extra orthogonal component from their difference. Thus $|a_x|_c$ is minimized (for fixed $N$) when the two representations coincide. In summary, the number $N$ is embedded as a single object $a_x$ that can be “read” in multiple ways: if one projects $a_x$ onto the subalgebra generated by $u$, one reads off the decimal digits $d_i$; if one projects onto the subalgebra generated by $v$, one reads off the binary digits $b_j$. The equality of these projections is enforced by the construction and by the coherence norm principle, so any observer (no matter which base they choose to interpret the object) will extract the same number $N$. This showcases UOR’s ability to unify different symbolic representations of the same entity.

    \item \textbf{Embedding Algebraic Structures (Fields and Groups):} The Clifford algebra is rich enough to naturally contain examples of various algebraic structures, which the UOR framework can exploit to embed those structures entirely. For instance, consider the field of complex numbers $\mathbb{C}$. It can be embedded in a real Clifford algebra of dimension 2: take $M$ two-dimensional Euclidean (so $\Cl(2,0)$ has basis $\{1, e_1, e_2, e_1 e_2\}$). In $\Cl(2,0)$, the element $I := e_1 e_2$ behaves like the imaginary unit: indeed $I^2 = e_1 e_2 e_1 e_2 = - e_1 e_1 e_2 e_2 = -1$ (because $e_1 e_2 = - e_2 e_1$ and $e_1^2 = e_2^2 = 1$ for an orthonormal basis in Euclidean signature). Therefore the subalgebra $\{a + bI : a,b \in \mathbb{R}\}$ is isomorphic to $\mathbb{C}$. In the UOR setting, if one wants to embed a complex number $z = a + ib$, one can choose a point $x$ and take $a_x = a\,1_x + b\,I_x \in \Cl(T_xM)$ (with $I_x$ the pseudoscalar at $x$). This $a_x$ has a grade-0 part $a$ and grade-2 part $b\,e_1 e_2$, and the coherence norm sees it as just one object with components $(a, b)$ in that subspace. The field operations of $\mathbb{C}$ correspond exactly to the restrictions of Clifford addition and multiplication to this subalgebra. So adding or multiplying two complex numbers embedded in $\Cl(2,0)$ yields the embedded result. Another example: the quaternions $\mathbb{H}$ can be embedded in $\Cl(3,0)$ (the algebra of $\mathbb{R}^3$). If $e_1, e_2, e_3$ are orthonormal basis of $T_xM \cong \mathbb{R}^3$, then one can verify that $e_2 e_3$, $e_3 e_1$, and $e_1 e_2$ behave like the fundamental quaternion units $i,j,k$ (indeed $(e_2 e_3)^2 = e_2 (e_3 e_2) e_3 = - e_2 e_2 e_3 e_3 = -1$, and similar for cyclic permutations, with the correct multiplication table for $i,j,k$). Thus any quaternion $q = w + x i + y j + z k$ can be represented as $a_x = w\,1_x + x\,(e_2 e_3) + y\,(e_3 e_1) + z\,(e_1 e_2)$ in the Clifford algebra. This is again a single element in $\Cl(T_xM)$ that encodes the quaternion’s components in the 0-grade and 2-grade parts. The norm $|\cdot|_c$ on this representation corresponds to the usual quaternion norm $\sqrt{w^2+x^2+y^2+z^2}$. Beyond fields, group structures can also be realized. Many groups of interest appear as multiplicative subgroups of Clifford algebras. For example, the set $\{\pm 1, \pm e_1 e_2, \pm e_2 e_3, \pm e_3 e_1\}$ in $\Cl(3,0)$ (with $e_i$ as above) forms a group isomorphic to the quaternion group of order 8. More generally, any orthogonal transformation on $T_xM$ can be represented by an element of the Clifford algebra (a so-called \emph{spinor} or \emph{versor}); the spin group $\mathrm{Spin}(n)$ is by definition a subgroup of $\Cl(n,0)$. Thus one can embed every rotation (an element of $SO(n)$) as two choices of elements $\pm A_x \in \Cl(T_xM)$ such that $a_x \mapsto A_x a_x A_x^{-1}$ is the rotation of that multivector. In the UOR context, this means we can choose to represent transformations either as extrinsic operations (the group action via $\Phi(g)$ on the bundle) or as intrinsic algebraic elements acting by conjugation. Both viewpoints exist within the same framework, which is another form of “universal reference”: not only objects, but transformations can be encoded as objects too. Summarizing, any finite group can be embedded in some Clifford algebra (via a representation or Cayley table encoding), any ring or field can be embedded (via matrices or direct algebra embedding if it’s a subfield of $\mathbb{R}$ or $\mathbb{C}$, etc.), and thus the UOR can encompass structures from number systems to symmetry groups. The coherence norm in these cases ensures that if a single abstract group element has multiple representations in the algebra (perhaps through different factorization or different coordinate definitions), those must coincide for the representation to be minimal-norm, thereby guaranteeing the consistency of the group law in the embedded form.

\end{itemize}

\end{document}
