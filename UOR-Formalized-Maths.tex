\documentclass[12pt]{article}
\usepackage{amsmath, amssymb, amsthm}
\usepackage{hyperref}
\usepackage{geometry}
\geometry{a4paper, margin=1in}

\title{Mathematical Formalization of Universal Object Reference (UOR) -- Part I}
\author{}
\date{}

\begin{document}

\maketitle

\section{Introduction and Motivation}
\begin{itemize}
    \item \textbf{Purpose:} To create a universal coordinate system where every definable entity (physical, mathematical, computational, or cognitive) is represented as an "object."
    \item \textbf{Philosophical Underpinning:} Inspired by the Mathematical Universe Hypothesis and ontic structural realism.
\end{itemize}

\section{Axiomatic Foundation}

\subsection{Axiom 1: Reference Manifold (M, g)}
\begin{itemize}
    \item \textbf{Definition:} A smooth, connected, oriented Riemannian (or pseudo-Riemannian) manifold \( M \) equipped with a metric \( g \).
    \item \textbf{Role:} Provides the continuous "stage" or context in which objects reside.
    \item \textbf{Additional Structure:} Admits a \( \text{spin}^c \) structure to ensure the existence of a global Clifford bundle.
\end{itemize}

\subsection{Axiom 2: Clifford Algebra Fibers (\(\mathcal{C}_x\))}
\begin{itemize}
    \item \textbf{Definition:} For every point \( x \in M \), assign the Clifford algebra \( \mathrm{Cl}(T_x M, g_x) \) built from the tangent space \( T_x M \) and the metric \( g_x \).
    \item \textbf{Role:} Encodes local algebraic and geometric information (vectors, bivectors, etc.) that represent the internal structure of objects.
\end{itemize}

\subsection{Axiom 3: Symmetry Group Action (G, \(\Phi\))}
\begin{itemize}
    \item \textbf{Definition:} A Lie group \( G \) acts smoothly on \( M \) by isometries and lifts to a fiberwise automorphism \( \Phi(g) \) on the Clifford bundle.
    \item \textbf{Role:} Enforces symmetry invariance; any change of coordinates or "observer" is handled by \( G \) and its induced actions.
\end{itemize}

\subsection{Axiom 4: Coherence Inner Product and Norm (\(|\cdot|_c\))}
\begin{itemize}
    \item \textbf{Definition:} Each Clifford fiber is equipped with an invariant inner product \( \langle \cdot, \cdot \rangle_c \), from which one defines a coherence norm \( |a_x|_c = \sqrt{\langle a_x, a_x \rangle_c} \).
    \item \textbf{Role:} Measures the "consistency" or "resource allocation" of an object’s representation and enforces that multiple representations (e.g., different base expansions of a number) agree.
\end{itemize}

\subsection{Axiom 5: Base Decomposition}
\begin{itemize}
    \item \textbf{Definition:} Every element \( a_x \in \mathrm{Cl}(T_x M) \) admits a unique decomposition into homogeneous (grade) components:
    \[
    a_x = a_x^{(0)} + a_x^{(1)} + \cdots + a_x^{(n)}
    \]
    Each component corresponds to scalar, vector, bivector parts, etc.
    \item \textbf{Role:} Provides explicit "coordinate" representations for objects, making the UOR a universal addressing system.
\end{itemize}

\section{Core UOR Structure}
\subsection{Formal Definition}
\begin{itemize}
    \item The Universal Object Reference is defined as the 6-tuple:
    \[
    \mathcal{U} = \bigl(M, g, \mathcal{C}, G, \Phi, |\cdot|_c \bigr)
    \]
    \item \textbf{Object Reference:} An object is represented as a pair \( (x, a_x) \) with \( x \in M \) and \( a_x \in \mathcal{C}_x \).
\end{itemize}

\subsection{Key Features}
\begin{itemize}
    \item All objects have an unambiguous "address" in \( M \) and an algebraic "content" given by their Clifford expansion.
    \item The symmetry group \( G \) acts transitively, ensuring that the UOR is universal and coordinate invariant.
\end{itemize}

\end{document}
